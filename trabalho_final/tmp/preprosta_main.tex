

%
% Exemplo de utiliza??o do modelo pgeeltex.
%
% por Miguel Moreto, 2011
%  
% Use por sua conta e risco.
% 
% Defini??o da classe a ser utilizada:
\documentclass[oneside,normaltoc,espacoduplo,PGTEXdissertacao]{pgeeltex}
% O significado das op??es ? apresentado no manual.

% Utilize tamb?m a op??o PGTEXdraft caso voc? queira gerar uma vers?o p ara a
% banca examinadora em tamanho A4 e para impress?o em somente u m dos lados da
% folha.
% Sempre mantenha seu Tex / Miktex atualizado.

\usepackage[brazil]{babel}
\usepackage[utf8]{inputenc}
\usepackage[pt,showend]{programma}
\usepackage{longtable}


% A linha abaixo chama o pacote hyperref, configurado para usar o programa 
% dvipdfm.exe na convers?o do .dvi para o .pdf. Se voc? utiliza outra forma de
% gerar o pdf, altere a linha abaixo com a op??o adequada. Veja o manual do 
% hyperref para saber as op??es dispon?veis e outras configura??es do pacote.

% Sugest?o: Gere o pdf utilizando o dvipdfm.exe que converte direto
% de dvi para pdf. Caso contr?rio, configure adequadamente o pacote hyperref.
\usepackage[dvips,linktocpage=true,pdfauthor={Alexander de Almeida Pinto},
pdftitle={Utilização das metaheurésticas GRASP e ILS aditivado com programção
linear para resolução do problema de construção de trilhos de aeronaves},
pdfsubject={PCTA},pdfcreator={LaTeX e Dvips}]{hyperref}
\usepackage[alf,abnt-emphasize=bf,abnt-etal-text=it,bibjustif,abnt-etal-cite=2,abnt-etal-list=0,abnt-url-package=hyperref,recuo=0.5cm]{abntcite}
 
% Se voc? quiser utilizar o estilo de cita??o num?rica, ex. [1], [2], comente a linha acima e use o c?digo que est? comentado nas 3 linhas abaixo:
  %\usepackage[num,bibjustif,recuo=1cm]{abntcite} 
  %\usepackage{colchetesrefs} % Coloca colchetes na lista de refer?ncias ao final do documento.
  %\citebrackets{[}{]}
% N?o esque?a do comando \bibliographystyle{abnt-num} antes the chamar o seu arquivo .bib. (\bibliography) ao final do documento para que a lista de refer?ncias tamb?m esteja no formato num?rico.

\usepackage{nomencl} % Pacote necess?rio para a lista de siglas. 
% O comando a seguir diz ao Latex para salvar as siglas em um arquivo separado:
\fazlistasiglas[Lista de Abreviaturas e Siglas] % Par?metro opcional ? o t?tulo da lista.

\usepackage[font={bf,small}]{caption}

\usepackage{graphicx}



%\renewcommand{\familydefault}{cmr}
% A seguinte linha evita algumas mensagens de warning por parte do hyperref.
% Comentar se o pacote hyperref n?o estiver sendo utilizado.
\pdfstringdefDisableCommands{\edef\uppercase{}}


% Os comandos a seguir ajustam o posicionamento dos floats (figures e tables).
\renewcommand{\floatpagefraction}{0.90}
\renewcommand{\topfraction}{0.95}
\renewcommand{\bottomfraction}{0.95}
\renewcommand{\textfraction}{0.05}
\setlength{\intextsep}{5pt}
\setlength{\textfloatsep}{5pt}
\setlength{\floatsep}{5pt}
%%%%%%%%% In?cio das defini??es iniciais %%%%%%%%%


\autor{Alexander de Almeida Pinto}
\titulo{Utilização das metaheurísticas GRASP e ILS com busca local exata para resolução do problema de construção de trilhos de
aeronaves}
\titlePGTEX{Titulo em ingles}
% T?tulo do documento \titlePGTEX{\LaTeX ~Thesis/Dissertation Example Using the \pgeeltex} % T?tulo do documento, em ingl?s.

% Nos 3 comandos abaixo o par?metro opcional entre [] pode ser preenchido para o caso de
% alguma dessas pessoas ser mulher. Se n?o especificado, vale o valor default.
%  Valores default: Orientador, Co-orientador e Coordenador.
\orientadorPGTEX[Orientador]{Dr.}{Lucídio dos Anjos Formiga Cabral}

%\coorientadorPGTEX[Co-orientadora]{Grau}{Nome do co-orientador}
\coordenadorPGTEX[Coordenadora]{Dra.}{Tatiana Aires Tavares} % Coordenador do PPGEEL

\areaconcentracaoPGTEX{Computação Distribuída} 
% ?rea de concentra??o 
\concentrationareaPGTEX{Distributed Computing} %
% ?rea de concentra??o em ingl?s

\palavraschavePGTEX{Transporte, PCTA, Metaheurística, Método Exato,GRASP, Rotas e Aeronaves}
% Palavras chave 
\keywordsPGTEX{Transportation, ARP,  metaheuristic, Exact Method, GRASP,Aircraft Routing}
% Palavras chave em ingl?s

% Instru??es para banca examinadora:
%   -> O primeiro membro da banca, definido pelo comando \bancaAPGTEX ? o presidente da banca
%      (provavelmente o orientador).
%   -> O segundo membro ser? automaticamente o Co-orientador se ele existir e for definido pelo
%      comando \coorientadorPGTEX
%   -> Os demais membros da banca s?o definidos pelos comandos \bancaBPGTEX, \bancaCPGTEX, \bancaDPGTEX e
%      \bancaEPGTEX.
%   -> Deixando os parametros dos comandos dos membros da banca vazios (exemplo \bancaAPGTEX{}), far?
%      com que as respectivas linhas de assinatura n?o apare?am.
%   -> Os comandos suportam um par?metro adicional, entre colchetes, onde ? colocado o grau de titula??o
%      do respectivo membro da banca. Se este par?metro adicional n?o for utilizado, ent?o o valor padr?o
%      "Doutor" ? atribuido.

\bancaAPGTEX[Dr., UFPB]{Roberto Quirino do Nascimento}
\bancaBPGTEX[Dra., UFF]{Lúcia Maria de Assumpção Drummond}
%\bancaCPGTEX{} % Como exemplo, o membro D da banca n?o vai aparecer por estar vazio.
%\bancaDPGTEX{}
%\bancaFPGTEX[Doutor, MIT]{Membro F da banca}
%\bancaGPGTEX[Doutor]{Membro G da banca}

\local{João Pessoa} 

\mesPGTEX[Março]{Mar} % Mes de publica??o
\anoPGTEX{2012} % Ano
\datadefesaPGTEX{23/03/2012} % Data da defesa.

% Dados para confec??o da capa de um trabalho de disciplina.
% Nescess?rios somente se a op??o de classe PGTEXtrabalho for utilizada.
\labPGTEX{LABSPOT}
\discipPGTEX{Disciplina 1}

%%%%%%%%% Fim das defini??es iniciais %%%%%%%%%

\begin{document} % In?cio do documento.

% Monta a capa (no novo modelo da BU, a capa ? feita separadamente, o texto inicia com a folha de rosto):
%\capaPGTEX

% Monta a folha de rosto:
\folhaderostoPGTEX
% Monta a folha de aprova??o:
\folhadeaprovacaoPGTEX

\doublespacing
% DEDICAT?RIA
%
% Est?o dispon?veis dois tipos de dedicat?ria:
\begin{dedicatoriaPGTEX}
Dedico este trabalho a minha família que me ajudou em todos os momentos
que precisei.
\end{dedicatoriaPGTEX}

% e / ou:

  

% EPIGRAFE (opcional).
% Parametro obrigatorio entre {} ? o autor da epigrafe.
\begin{epigrafePGTEX}{Albert Einstein} 
Se você continua vivo é porque ainda não chegou aonde devia.
\end{epigrafePGTEX}


% AGRADECIMENTOS
\begin{agradecimentosPGTEX} 
Fazer um agradecimento que inclua todos que me ajudaram a construir
essa dissertação não é uma tarefa fácil, principalmente por se correr
o risco de esquecer de mencionar alguém.

No âmbito acadêmico devo agradecer principalmente a excelência profissional do
Dr. Lucídio dos Anjos Formiga Cabral que conferiu prestígio e valor a meu
trabalho de mestrado além de ser um amigo de grande valia. Também gostaria de
fazer um agradecimento especial ao Roberto Pontes que teve uma grande
participação nesse projeto com idéias que foram resgatadas do seu trabalho
relizado na COPPE/RioSul. E também aos demais professores que me acompanharam
durante a graduação e o mestrado, me ensinando conceitos importantíssimos.

Agradeço também a meus amigos que, de uma forma ou de outra,
contribuíram com sua amizade e com sugestões efetivas para a realização deste
trabalho, principalmente a Daniel Gonçalves, Gilberto Farias e Rennan Toscano. 
Gostaria de expressar minha profunda gratidão.

Não poderia esquecer de meus pais, que me deram contínuo apoio em todos esses
anos, ensinando-me os valores que irei levar durante toda minha vida e que
acreditaram na minha capacidade de realização tornando-se assim os elementos
propulsores desta dissertação. 

Agradeço também ao Conselho Nacional de Desenvolvimento Científico e
Tecnológico, CNPq, pela bolsa concedida durante os anos do curso.

\end{agradecimentosPGTEX}

% RESUMO:
\begin{resumoPGTEX}
\doublespacing
Os problemas operacionais cresceram muito em complexidade nos
últimos tempos, o que tem acentuado a necessidade do desenvolvimento de
técnicas que possam agilizar os processos de tomada de decisão. 

%Empresas que
%não utilizam sistemas computadorizados com essa finalidade tem perdido espaço
%entre seus concorrentes.

Este trabalho trata da etapa de geração dos trilhos de aeronaves, ou seja, o
sequenciamento de voos de cada aeronave. O objetivo aqui é minimizar o número de
aeronaves necessárias para operar uma dada malha de voos.


 
 %A construção de trilhos de aeronaves é considerado um dos principais problemas
 %da indústria aeronáutica e se refere ao sequênciamento dos voos de uma
 %companhia aérea de forma que o menor número de aeronaves seja necessário para
 %operá-los.
 
 Esse problema possui é combinatoriamente explosivo e a sua resolução fica mais
 difícil à medida que a quantidade de voos envolvidos cresce. Pequenas
 modificações nos horários de partida desses voos, ou o acréscimo de algum voo
 de resposicionamento entre dois aeroportos próximos podem gerar soluções de
 melhor qualidade.
 
 Nós apresentamos um algoritmo híbrido baseado na
 metaheurística GRASP, com a utilização do ILS e de programação inteira na
 busca local.
 %A sua utilização é indicada para resolução de problemas de larga escala, pois
 % nesse caso fica inviável a aplicação de um algoritmo puramente
 % exato que poderia levar anos para realizar a tarefa. 
 Os resultados tem mostrado que essa abordagem é capaz de
 gerar boas soluções.
 
 \\
 
 \end{resumoPGTEX}

% ABSTRACT
\begin{abstractPGTEX}
\doublespacing

Operational research problems has growing in complexity in the last years, this
has accentuated the necessity to develop of techniques witch can accelerate the
process of decision making.

This work covers the step of aircraft rotation problem, i.e., the sequencing of
flights for each aircraft. The goal here is minimize the number of aircraft
required to operate a given network of flights.

This problem is combinatorial and it resolution is more difficult when the
number of involved flights grows. However small changes in departure time, or
the addition of a repositioning flight between two nearby airports can reduce
the cost of solutions.

We present a hybrid algorithm based on the metaheuristic GRASP, using the ILS
and integer programing in the local search.

The results has shown which this approach can generate good solutions.

%Operational problems has grown in complexity nowadays, what has made
%necessary the development of techniques wich can accelerate the decision
%making. Companies that are not using computerized systems for this purpose has
%wasted space among its competitors.

%The aircraft rotation problem is considered one key problem in the
%aeronautical industry and refers to the scheduling of flights from an
%airline such that the fewest aircraft is required to operate them. This problem
%is combinatorial and its resolution is more difficult when the number of
%involved flights grows. However small changes in departure times, or the
%addition of a repositioning flight between two nearby airports can
%reduce the cost of solutions.

%We present a hybrid algorithm based on the metaheuristic GRASP, using the ILS
%and integer programing in the local search. This algorithm is intended for
%solving large-scale problems, because in this case is not feasible to apply an
%pure exact algorithm that could take years to perform the task. 
%The results needed short time to get good solutions.

\end{abstractPGTEX}


\listadefiguras

\listadetabelas

\listadesiglas

%\listadesimbolos

\sumario
  
\chapter{Introdução}
  
 A aviação é o principal meio de transporte de pessoas e de mercadorias capaz
 de atravessar grandes distâncias além de ser fundamental para o crescimento
 da economia e do turismo mundial contribuindo assim para a melhora da
 qualidade de vida das pessoas proporcionando lazer e experiências com outras
 culturas. O uso da aviação comercial tem crescido significativamente nas
 últimas décadas e a previsão é que esse aumento seja cada vez maior, pelo
 menos no Brasil, que está fazendo grandes investimento para ampliar e
 melhorar a infraestrutura aeroportuária com a finalidade de atender a grande
 demanda que é esperada para a Copa do Mundo de 2014 e para as Olimpíadas de
 2016. Além disso a economia do país está crescendo e provocando um aumento da
 renda e da qualidade de vida, incentivando as pessoas a viajar e explorar novas
 oportunidades.

%Pode se falar também do aumento da confiança no setor aereo que está sendo
% associado a uma forma segura de viajar.
  	
Esse ambiente favorável tem provocado o aumento do número de companhias
aéreas, fazendo com que ocorra uma maior competição, isso faz com que acentua-se
a necessidade de otimizar a utilização dos recursos disponíveis, reduzindo os
custos para poder continuar a crescer e oferecer tarifas mais competitivas.
Porém, os problemas presentes na indústria aeronáutica são complexos envolvendo
múltiplas decisões conflitantes que precisam ser otimizadas em conjunto.
Diversas técnicas são desenvolvidas e utilizadas para tentar melhorar o
planejamento e a operação das empresas aéreas. Muitas dessas técnicas estão
disponíveis na literatura científica, nos campos da pesquisa operacional e da
matemática. Normalmente essas técnicas são modeladas para funcionar em sistemas
computadorizados de alta capacidade, com a finalidade de automatizar, ou pelo
menos auxiliar na tomada de decisões. Essas técnicas se tornam
necessárias à medida que a empresa aérea cresce e o processo de tomada de
decisão, baseada nos julgamentos individuais e nas experiências se torna mais
difícil \cite{ahmed2009}.
  	
  	
Os principais problemas relacionados dizem respeito ao planejamento, envolvendo
a criação de linhas de trabalho (sequência de voos) tanto para as aeronaves
quanto para a tripulação. O objetivo costuma ser a minimização dos custos operacionais ou a
maximização dos rendimentos. Custos operacionais consistem, por exemplo, nos
custos envolvidos com combustíveis, óleo e taxas de aterrissagem. Também pode
ser levado em consideração a perda de rendimentos com a utilização de aeronaves
com menos assentos do que a demanda de passageiros, provocado pelo mal
dimensionamento da demanda. Fatores como o bem estar dos passageiros também pode
ser levado em consideração, provocando por exemplo uma redução na quantidade de
conexões e de escalas.
	
Para que seja possível ter uma visão geral do contexto é necessário descrever os
problemas que estão associados com a construção de trilhos de aeronaves. O
primeiro dos problemas que será mencionado aqui é a modelagem de mercado que tem
como objetivo fazer o levantamento da quantidade de passageiros que tem
interesse em viajar entre as localidades levando em consideração o espaço de
tempo desejado. Adicionalmente pode-se criar demandas através da criação de
novas rotas com o auxilio de propagandas e promoções. Essa demanda normalmente
apresenta uma grande variedade dependendo da época do ano e o levantamento desses dados de forma errada pode
causar grandes prejuízos. É importante categorizar o tipo de passageiro para
poder fazer escolhas futuras que não levem em consideração apenas fatores
quantitativos.

Posterior a modelagem de mercado tem-se a necessidade de definir qual
equipamento (frota) irá operar cada trecho definido pela
demanda \cite{pimentel2005}, esse problema é conhecido como a atribuição de
frota (\textit{Fleet Assigment Problem}) . A definição de um equipamento errado
pode provocar a subutilização da aeronave o que provocaria prejuízos ou a
superutilização que provocaria a perda de clientes para a
concorrência. Ao final dessa etapa irão ser formados
conjuntos de voos que deverão ser operados por tipos de aeronaves específicas. É a partir desses conjuntos que deverá ser feito a
construção dos trilhos das aeronaves.

Um trilho de aeronave é a sequência de voos que uma aeronave deve operar em um
determinado período, e o problema de construção desses trilhos 
(\textit{Aircraft Rotation Problem}) tem como principal objetivo a
redução do número de aeronaves necessárias para operar todos os voos e como
objetivo secundário fazer o menor número de modificações no planejamento inicial desses
voos.

O problema seguinte utiliza das sequências de voos das aeronaves para construir
o melhor conjunto de \textit{pairings}\footnote{Pairing é o conjunto de voos que
pode ser operados por uma tripulação sem que seja violada qualquer regra da
legislação vigente e que ao final do último voo o tripulante esteja de volta a
sua cidade base.} de forma que cada voo seja coberto por pelo menos um pairing.
Gastos com alojamentos, alimentação, transporte em terra e
\textit{deadheads}\footnote{Deadhead é o voo que o tripulante viaja sem
trabalhar, com a finalidade de transporte para outra localidade normalmente para sua base ou para suprir uma nova demanda, o deadhead pode ser operado por aviões de outras companhias, nesse caso o custo da viagem é maior.} devem ser levados em consideração. A partir desse conjunto é feito a escala dos
tripulantes (\textit{Crew Scheduling Problem}) que atribui os pairings a
tripulação acrescentando as atividades de solo, tais como \textit{Call
Time}\footnote{Call time é o tempo que a tripulação tem para se apresentar a
companhia aérea antes de iniciar de fato seu turno de trabalho.},
\textit{Stand-by duties}\footnote{Stand-by duties
são turnos em que o tripulante fica a disposição da companhia aérea afim de
suprir possíveis eventualidades.} e os dias de descanso. O objetivo dessa última
etapa é fazer uma distribuição da forma mais justa possível, tentando balancear
a quantidade de trabalho (horas a serem voadas) entre os tripulantes e também
tentar cumprir suas preferências sem violar nenhuma restrição da
legislação trabalhista em vigor.
   

Esse trabalho mostra uma forma eficiente de resolver o Problema de Construção
de Trilhos de Aeronaves (PCTA) 
%\abbrev{PCTA}{Problema de Construção de Trilhos de Aeronaves} 
que também é conhecido na literatura como Aircraft Rotation
Problem (ARP) 
%\abbrev{ARP}{Aircraft Rotation Problem}
. O PCTA é um dos principais problemas presentes na industria da aviação e seu objetivo
é fazer o sequenciamento dos voos de cada frota da companhia de forma que seja
possível operá-las com o menor número de aeronaves possíveis \cite{abiliolivro}
bem como efetuar a menor modificação possível no planejamento inicial dos voos. Cada sequência de voos recebe o nome de trilho de aeronave e o conjunto desses
trilhos é denominado de malha aérea. 

Para resolver o PCTA foram aplicadas metaheurísticas combinadas com
método exato de \textit{branch-and-bound} através da
resolução de um modelo matemático de programação linear inteira simplificado.
Essa estratégia foi escolhida pois com os computadores atuais não é possível obter resultados apenas com a aplicação do modelo, que retorna
o resultado ótimo, pois o tempo necessário para resolver instâncias semanais do
PCTA já leva um tempo inviável. A utilização apenas de metaheurística foi levada
em consideração no início, porém nos resultados práticos a qualidade da solução
se mostrou muito aquém da desejada. Dessa forma surgiu a idéia de juntar as
duas técnicas, com a finalidade de obter a convergência do método exato e a
velocidade da metaheurística.

Na literatura tem-se observado um crescimento no número de trabalhos que se
utilizam de metaheurísticas híbridas como método de resolução de problemas
complexos de otimização combinatória apresentando soluções de alta qualidade
%%%%%%%%%%%%%%%%% (ACRESCENTAR CITACOES AQUI) <<<<<<<<<<<<< ------------
. Esse fato também aconteceu na resolução do PCTA.



%Para resolver o PCTA, devemos estar cientes de algumas restrições que envolvem
%tempo e espaço. Por exemplo, um voo não pode iniciar antes da chegada do voo
%que lhe antecede, nem de um local diferente da cidade de destino deste voo
%antecessor, esses exemplos são denominados respectivamente de restrições
%temporais e geográficas do problema. Há também a restrição de que um voo deve
%permanecer em solo, entre conexões, por um período de tempo que seja suficiente
%para fazer a troca de passageiros e abastecimento da aeronave e quando for o
%caso para a mudança da tripulação, esse tempo varia de acordo com o aeroporto e
%com o tipo de aeronave.

%Existe também a restrição de consistência que está presente em instâncias que
%possuem frequência. Nessas instâncias um voo pode aparecer em diversos dias.
%Dessa forma deve-se garantir que o horário de partida desse voo seja o mesmo em
%todos os dias que ele ocorrer, ou seja, caso alguma modificação seja feita no
%horário de partida sugerido desse voo, em um dos dias, então todos os outros
%dias da frequência também devem ser modificados.

%Outro aspecto importante diz respeito às restrições de manutenção. Sabe-se que
%um avião deve ter checagens periódicas. Oportunidades de realizar essas tarefas
%ocorrem apenas em algumas conexões potencialmente disponíveis. Como
%consequência, uma sequência de voos deve ser construída de forma que essas
%restrições não sejam violadas. A fim de incorporar essas restrições facilmente
%ao nosso framework, assumimos que as rotações são designadas a tipos não
%específicos de aeronave. Dessa forma, se uma aeronave tem necessidade de
%manutenção, um voo especial é criado com origem e destino na base de manutenção
%escolhida e com a sua duração exatamente igual ao tempo de manutenção 
%\cite{pontes2002}.

%	Vale ressaltar ainda que na resolução do PCTA deve-se levar em consideração as
%	particularidades especificas de cada companhia aérea como o número de aviões
%disponíveis na frota, o atraso máximo permitido nos voos, a quantidade máxima
%de voos que podem sofrer atraso, o número máximo de voos que podem ser
%cancelados, o número máximo de voos de reposicionamento que podem ser criados
%entre outros.
	
%De uma maneira geral, o principal objetivo do PCTA é a minimização do número de
%trilhos seguido da minimização do custo total dos trilhos gerados. Esse custo
%pode envolver diversos componentes, sendo o tempo médio diário de utilização
%das aeronaves um dos mais importantes \citep{abiliolivro}.
 
%###################################### 
%Aqui já nao descreve mais o problema.
%######################################
  
Nos dias de hoje, com o avanço da tecnologia e o aumento da competitividade,
desenvolver soluções com melhor qualidade acaba se tornando um fator decisivo
para a permanência no mercado, tornando-se então necessário a obtenção de
soluções de forma mais rápida, mais barata e utilizando menos recursos. Por
causa da dificuldade que é inerente a essa classe de problemas a qualidade das
soluções obtidas manualmente são muito inferiores em relação à melhor solução
possível. Outra característica que reforça a necessidade da obtenção de melhores soluções é o
aumento do tamanho e da complexidade das instâncias trabalhadas. A partir desse
cenário pode-se perceber a necessidade de utilização de técnicas de otimização.

  
%As metaheurísticas podem ser definidas como um conjunto de procedimentos de
%caráter geral, com capacidade de guiar o procedimento de busca, tornando-o
%capaz de escapar de ótimos locais. Elas têm como objetivo, encontrar uma
%solução tão próxima quanto possível da solução ótima do problema com baixo
%esforço computacional. Em geral, as metaheurísticas são bastante utilizadas na
%resolução de problemas de otimização. Esses problemas, também conhecidos como
%problemas NP-difíceis\abbrev{NP}{Non-Polinomial}, podem ser definidos como um
%conjunto de problemas para os quais ainda não existe um algoritmo que os
%resolvam de forma exata e em tempo polinomial \citep{maritan2009}. Para esses
%tipos de problemas o uso de métodos exatos é bastante restrito, uma vez que o
%esforço computacional para encontrar uma solução exata em instâncias reais é
%consideravelmente alto. Na prática, no entanto, é suficiente encontrar uma
%solução melhor que a utilizada no ambiente operacional.


%#############################################
%{\large >>>> Descrever em que tem se baseado a solução desse problema na literatura <<<<}
%Essa parte deveria ir para a revisão da literatura.
%#############################################

Em relação ao PCTA a literatura pesquisada \cite{ahmed2009} \cite{arguelo1997}
\cite{cordeau2001} \cite{mohamed2011} \cite{abiliolivro} mostra uma grande quantidade de
tentativas de resolver o problema utilizando modelagens matemáticas, que apesar
de garantir a solução ótima não se mostra viável para resolver grandes
instâncias. Alguns trabalhos mostram a similaridade desse problema com o
problema do caixeiro viajante assimétrico \cite{clarke97}. E outros resolvem
uma parte do problema utilizando metaheurísticas \cite{arguelo1997}. Além disso,
a pesquisa constatou que a literatura acerca do PCTA não disponibiliza as
instâncias que foram trabalhadas dificultando assim a comparação dos
resultados obtidos com esses trabalhos. Logo um dos objetivos desse trabalho é
disponibilizar um conjunto de instâncias.

%#############################################
%{\large	>>>> Fazer uma breve descrição de programação linear, maiores detalhes serão dados na fundamentação teórica. <<<<	}
%#############################################

Atualmente existem diversas fontes na qual se podem obter instâncias para
problemas de otimização combinatória sendo uma das mais conhecidas a
OR-Library \footnote{ A OR-Library pode ser acessado em
http://people.brunel.ac.uk/~mastjjb/jeb/info.html} que foi descrito
inicialmente em J.E.Beasley \cite{orlibrary} permitindo o acesso a centenas de
conjuntos de instâncias a partir da Internet.
  
Apesar da existência dessas entidades apenas uma instância, a da Rio Sul, foi
encontrada em um relatório técnico da UFPB. Com a finalidade de ter outra
instância, foi feito o  na web dos voos operados pela
compahia denominada TAM (http://www.tam.com.br). Esse levantamento foi feito
através de uma pesquisa observativa fazendo-se o comparativo da malha operada
pela companhia, que pode ser visto na Figura \ref{fig:malhatam}, e as
possibilidades de voos diretos do equipamento AirBus Industrie A310, que formava
a maior frota da empresa.
	
\begin{figure}[h]
\centering
\caption{Rotas de voo da companhia aérea TAM \newline \mbox{Fonte:
(http://www.airlineroutemaps.com) }}

\label{fig:malhatam}
\includegraphics[scale=0.45]{./img/tam_brazilian_airlines}
\end{figure}
	
A malha da Rio Sul se refere a uma instância com voos de um dia de operação da
empresa contendo 107 voos, onde na prática era operado com 20 trilhos
obtidos pela montagem manual de um experiente operador \cite{pontes2002}. A
malha da TAM, que foi obtida manualmente, é formada por 241 voos possuindo
uma grande quantidade de ligações entre os 31 aeroportos envolvidos, dessa
forma essa instância  obteve um maior grau de complexidade.

Essas instâncias foram estendidas para o período de uma semana para testar o
comportamento do algoritmo e do solver, onde a instância da Rio Sul
estendida apresentou 749 voos e a da Tam estendida ficou com 1687 voos.
	
Levou-se em consideração apenas instâncias de companhias aéreas brasileiras pois
as grandes empresas globais apresentam uma malha com característica
\textit{hub-and-spoke}, como pode ser visto na Figura \ref{fig:hubandspoke}. Uma
malha é caracterizada como sendo do tipo \textit{hub-and-spoke} se ela
apresentar uma grande concen tração de voos em poucos aeroportos e
adicionalmente se um voo tem ligação com um \textit{hub} ele não poderá ter
ligação com outros \textit{hubs} a não ser que ele também seja um \textit{hub}.
Instâncias com essa caractéristicas são mais fáceis de serem resolvidas pois
não apresentam a caractéristica explosiva no nível presente nas malhas aéreas brasileiras.
	
\begin{figure}[ht]
\caption{Malha hub-and-spoke. \newline \mbox{Fonte: (Própria)}}
\label{fig:hubandspoke}
\includegraphics[scale=0.35]{./img/hubandspoke}
\end{figure}
 	 	
	
Não foi possível obter mais instâncias reais pois a comunicação entre essas
empresas e a acadêmia é fraca. Grande parte dessa falta de comunicação se deve
ao receio de revelar dados que podem vir a lhes prejudicar junto a concorrência.

As instâncias podem ser vistas nos anexos ao final desse trabalho.


\section {Objetivos do trabalho}

% Falar das consequências dos objetivos (em termo de economia e em relação a
% melhora do trafego aereo, no brasil, pois irá fazer com que as aeronáves
% fiquem mais tempo voando. O principal problema hoje é na falta de espaço no
% solo para pousar os avioes.

\subsection{Geral}
Tendo em vista os aspectos apresentados, o objetivo principal dessa dissertação
consiste no desenvolvimento de um método híbrido baseado nas metaheurísticas
GRASP e ILS e em programação linear inteira para a resolução do problema
construção de trilhos de aeronaves (PCTA) cobrindo todos os voos planejados com
o menor número de aeronaves bem como efetuando a menor mudança possível no
planejamento inicial desses voos. 

\subsection{Específicos}

Esse trabalho também visa a disponibilização de um conjunto de instâncias com
os resultados que foram obtidos. Isso irá permitir a sua comparação com
outros trabalhos no futuro.

O método proposto consegue acelerar a convergência da fase de busca local
através de uma busca exata em uma vizinhança restrita usando programação
linear inteira. Adicionalmente esse método permite escapar de minimos locais.

Para conseguir melhores resultados foi permitido, em alguns cenários, a
utilização de pequenas alterações no horário de partida sugerido dos voos bem
como a criação de voos de reposicionamento.

\section {Organização da proposta }

O restante desse trabalho está estruturado da seguinte forma:

\begin{itemize}

\item Capítulo 2: Apresenta a fundamentação sobre otimização, metaheurísticas
e programação linear inteira. A seção referente às metaheurísticas inicia
com uma descrição, seguida do detalhamento das metaheurísticas utilizadas no trabalho,
o GRASP e o ILS.

%Ao final da fundamentação teórica
%será feita uma revisão dos principais trabalhos relacionados ao presente
%trabalho que estão presentes na literatura.

\item Capítulo 3: Descreve o problema, explicando os conceitos que são
utilizados no trabalho.

\item Capítulo 4: Introduz o modelo matemático que foi desenvolvido.

\item Capítulo 5: Descreve o método proposto nesse trabalho, mostrando como foi
feita a integração das metaheurísticas e da programação linear inteira e também descreve os parâmetros e as restrições que foram utilizadas.

\item Capítulo 6: Apresenta os resultados obtidos com o algoritmo e dá
diretrizes de como utilizar o método proposto.
%\item Capítulo 8: Dá uma conclusão, indicando o que tem que ser melhorado e o que se espera ter para a finalização do trabalho.

\item No final é apresentado a bibliográfia e os anexos que contém um maior
detalhamento das instâncias e dos resultados obtidos.

\end{itemize}
\chapter{Fundamentação Teórica}

Nesse capítulo é feita a fundamentação dos principais assuntos presentes nesse
trabalho: a heurística construtiva, as metaheurísticas e a programação
linear. Nas seções seguintes são descritas os aspectos teóricos e os
principais métodos relacionados a esse trabalho.

\section{Heurísticas Construtivas}
As técnicas de resolução heurísticas se utilizam de processos intuitivos com a
finalidade de obter uma boa solução, a um custo computacional aceitável, ou
seja não garante a otimalidade de um problema. O objetivo é obter em um tempo
reduzido uma solução tão próxima quanto possível do ótimo global.
		
Uma heurística é dita construtiva quando a construção da solução se dá elemento
por elemento. A forma de escolha dos elementos variam de acordo com a
estratégia e a função de avaliação adotada, essa escolha deve levar em
consideração o benefício da inserção de cada elemento para a solução final,
escolhendo sempre o \emph{melhor} elemento em cada passo.
		
O Algoritmo \ref{alg:heurconsgulosa} mostra o pseudocódigo para a construção de
uma solução inicial para um problema de otimização que utiliza uma função
gulosa \emph{g(.)}. Nesta figura, \emph{$t_{melhor}$} indica o membro do
conjunto de elementos candidatos com o valor mais favorável da função de
avaliação \emph{g}, isto é, aquele que possui o menor valor de \emph{g} no caso
de o problema ser de minimização ou o maior valor de \emph{g} no caso de o
problema ser de maximização.


\begin{figure}[h]
\caption{Pseudocódigo da heurística de construção gulosa de uma solução
inicial. \newline \mbox{Fonte:
\cite{notasmarcone}} }\label{alg:heurconsgulosa}
\begin{programma}
\ALGORITHM{$ConstruçãoGulosa(g(.), s$)}
\STATE s \GETS $\emptyset$;
\STATE Inicialize o conjunto $C$ de candidatos;
\WHILE{$C \neq \emptyset$}
\STATE $g(t_{melhor}) = melhor\{g(t) \mid t \in C\}$;
\STATE $s \GETS s \cup \{t_{melhor}\}$;
\STATE Atualize o conjunto $C$ de elementos candidatos;
\ENDWHILE
\STATE\RETURN $s$;
\ENDALGORITHM
\end{programma}
\end{figure}		
		

Uma outra forma de obter uma solução inicial é escolhendo os elementos
candidatos aleatoriamente. Isto é, a cada passo, o elemento a ser inserido na
solução é aleatoriamente selecionado dentre o conjunto de elementos candidatos
ainda não selecionados. A grande vantagem desta metodologia reside na
simplicidade de implementação. Segundo testes empíricos , a desvantagem é a
baixa qualidade, em média, da solução final. Essa técnica é recomendada quando
a característica do problema torna mais fácil o refinamento do que a construção
de uma solução \cite{notasmarcone}.

O Algoritmo \ref{alg:heurconsaleatoria} mostra o pseudocódigo para a construção
de uma solução inicial aleatória para um problema de otimização.

\begin{figure}[h]
\caption{Pseudocódigo da heurística de construção aleatória de uma
solução inicial. \newline \mbox{Fonte:
\cite{notasmarcone}}}\label{alg:heurconsaleatoria}
\begin{programma}
\ALGORITHM{$ConstruçãoAleatória(g(.), s$)}
\STATE s \GETS $\emptyset$;
\STATE Inicialize o conjunto $C$ de candidatos;
\WHILE{$C \neq \emptyset$}
\STATE Escolha aleatoriamente $t_{escolhido} \in C$;
\STATE $s \GETS s \cup \{t_{escolhido}\}$;
\STATE Atualize o conjunto $C$ de elementos candidatos;
\ENDWHILE
\STATE\RETURN $s$;
\ENDALGORITHM
\end{programma}
\end{figure}

Para melhores resultados essa etapa deve ser seguida de um refinamento, pois a
solução, quando gerada aleatoriamente, não costuma ser de boa qualidade.

\section{Metaheurística}

A utilização de métodos exatos para a resolução de problemas reais envolvendo
otimização combinatória é restrito. Isso acontece pois com o aumento das
instâncias envolvidas, o número de soluções possíveis cresce exponencialmente,
fazendo com que as operações necessárias para a sua resolução não possa feita
com os computadores atuais em tempo viável.

Para contornar essa limitação e obter soluções para esses tipos de problemas,
os pesquisadores desenvolveram técnicas que são capazes de guiar o procedimento
de busca e assim encontrar boas soluções \cite{maritan2009}. Esses algoritmos,
denominados heurísticas, encontram essas soluções utilizando pouco recursos
computacionais, porém não garantem a solução ótima do problema \cite{dias2006}.
Na prática, geralmente, uma boa solução é suficiente, já que a tomada de
decisão tem que acontecer em um curto espaço de tempo.

As metaheurísticas são aplicadas a uma maior gama de problemas e surgiu para
suprir algumas deficiências das heurísticas. Para contornar essas deficiências,
foram desenvolvidas técnicas mais generalistas que foram denominadas de
metaheurísticas. As metaheurísticas podem ser definidas como sendo um método
heurístico para resolver de forma genérica problemas de otimização com a
capacidade de escapar de ótimos locais. A idéia utilizada, normalmente, é
obtida de algum evento natural como sistemas biológicos, da física, da
inteligência artificial entre outros.

As metaheurísticas podem explorar o espaço de soluções basicamente de duas
formas: as metaheurísticas de busca local e as metaheurísticas de busca
populacional. Nas metaheurísticas de busca local, o procedimento de busca
utiliza uma solução como ponto de partida em cada iteração. As metaheurísticas
GRASP, arrefecimento simulado (\textit{Simulated Annealing}), busca tabu e ILS
podem ser citadas como exemplos de metaheurísticas ponto-a-ponto. Nas metaheurísticas de
busca populacionais, soluções de boa qualidade são combinadas com o intuito de
produzir soluções melhores. Podemos citar como exemplo de métodos
populacionais, os algoritmos genéticos, colônia de formigas (\textit{Ant Colony
System}), núvem de particulas (\textit{Particle Swarm Optimization}) e etc
\cite{maritan2009}.

Nesse trabalho foram utilizados as metaheurísticas de busca local GRASP e ILS
de forma híbrida. As próximas seções descrevem essas metaheurísticas.

\subsection{GRASP}

Essa seção descreve a metaheurística GRASP (\textit{Greedy Randomized Adaptive
Search Procedure} - Procedimento de busca adaptativa gulosa e randômica), que
foi proposto por \cite{resende1995}, e cujos conceitos serão
utilizados na metodologia proposta para resolução do PCTA. A metaheurística
GRASP é um método iterativo do tipo \textit{multi-start} formado por duas
fases: uma fase de construção de uma solução e outra de busca local. A fase de
construção objetiva gerar uma solução viável para o problema proposto. E a fase
de busca local na qual se pesquisa  um ótimo local na vizinhança da solução
construída. A melhor solução encontrada, ao longo de todas as
iterações GRASP realizadas, é retornada.

O pseudo-código descrito no Algoritmo \ref{alg:grasp} ilustra um procedimento
GRASP para um problema de minimização. Na linha 1 o custo da função objetivo da
melhor solução encontrada é inicializada com $\infty$(infinito). A linha 2
repete o procedimento de construção e refinamento $GRASPMax$ vezes, por causa dessa
etapa que o GRASP é considerado \textit{multi-start}.

Na linha 3 e 4 são feitas respectivamente a construção e a busca local que são
representadas nos Algoritmos \ref{alg:graspcons} e \ref{alg:grasplocal} e serão
detalhadas mais adiante.

Nas linhas 5 à 8, se a solução obtida na busca local for melhor que a melhor
solução obtida até o momento ($f(s) < f{*}$) então são atualizadas
respectivamente a solução e o custo relativo a função objetivo da melhor
solução corrente. A linha 9 encerra as iterações do GRASP e a linha 10 retorna a
melhor solução obtida na execução do algoritmo.

\begin{figure}[h]
\caption{Pseudocódigo do procedimento GRASP. \newline \mbox{Fonte:
\cite{resende1995}}}\label{alg:grasp}
\begin{programma}
\ALGORITHM{GRASP($f(.), g(.), N(.), GRASPMax, s$)}
\STATE f{*} \GETS $\infty$;
\FOR{$1, 2, ..., GRASPMax$}
\STATE Construção($g(.), \alpha, s$);
\STATE BuscaLocal($f(.),N(.),s$);
\IF{$f(s) < f{*}$}
\STATE $s{*} \GETS s$;
\STATE $f{*} \GETS f(s)$;
\ENDIF
\ENDFOR
\STATE\RETURN $s{*}$;
\ENDALGORITHM
\end{programma}
\end{figure}

Na fase de construção uma solução é iterativamente construída, elemento por
elemento. A parte gulosa da função visa gerar uma solução factível. O
componente aleatório é incluído para explorar regiões diversas do espaço de
soluções e é uma das chaves da efetividade do GRASP.

A fase de construção do GRASP é baseada na construção de uma lista restrita de
candidatos (LCR). Essa lista contem os melhores candidatos que podem ser
adicionados a solução em um dado momento, a quantidade de elementos dessa lista
é regulada pelo $\alpha$ que é um dos parâmetros do GRASP. O $\alpha$ é
definido como sendo o nível de aleatoriedade da solução.

\begin{figure}[h]
\caption{Pseudocódigo do procedimento de construção do GRASP para um problema
de minimização. \newline \mbox{Fonte:
\cite{resende1995}}}\label{alg:graspcons}
\begin{programma}
\ALGORITHM{$Construção(g(.), \alpha,s$)}
\STATE s \GETS $\emptyset$;
\STATE Inicialize o conjunto $C$ de candidatos;
\WHILE{$C \neq \emptyset$}
\STATE $g(t_{min}) \GETS min\{g(t) \mid t \in C\}$;
\STATE $g(t_{max}) \GETS max\{g(t) \mid t \in C\}$;
\STATE $LCR \GETS \{t \in C \mid g(t) \leq g(t_{min}) + \alpha(g(t_{max}) - g(t_{min}))\}$;
\STATE Selecione aleatoriamente um elemento $t \in LCR$;
\STATE $s \GETS s \cup \{t\}$;
\STATE Atualize conjunto de candidatos;
\ENDWHILE
\STATE\RETURN $s$;
\ENDALGORITHM
\end{programma}
\end{figure}

Em cada iteração são selecionados todos os elementos que podem ser
inseridos na solução e então é formada uma lista de candidatos que é ordenada
segundo algum critério pré-determinado, no caso de um problema de
minimização a lista normalmente é ordenada de acordo com o acréscimo na função
objetivo que esse elemento acarretaria se fosse escolhido. A
heurística é dita adaptativa porque os benefícios associados com a escolha de
cada elemento são atualizados em cada iteração da fase de construção para
refletir as mudanças oriundas da seleção do elemento anterior. A componente
probabilística do procedimento reside no fato de que cada elemento é
selecionado de forma aleatória a partir de um subconjunto restrito formado
pelos melhores elementos que compõem a lista de candidatos. Este subconjunto
recebe o nome de lista de candidatos restrita (LCR). Esta técnica de escolha
permite que diferentes soluções sejam geradas em cada iteração GRASP
\cite{notasmarcone}. O valor do grau de aleatoriedade $\alpha$ se encontra
entre [0,1].

Um valor de $\alpha = 0$ faz com que o algoritmo gere soluções puramente
gulosas enquanto a escolha de um $\alpha = 1$ faz com que o algoritmo gere
soluções puramente aleatórias.
 
A construção do GRASP difere do Algoritmo \ref{alg:heurconsgulosa} por causa
das linhas 4 à 7. A linha 4 obtém o valor mínimo que será acrescentado a
solução final, dentre os candidatos possíveis e a linha 5 obtém o valor máximo.
A linha 6 forma a LCR com os elementos que tiverem o valor entre $g(t_{min}) +
\alpha(g(t_{max}) - g(t_{min}))$. Por fim a linha 7 seleciona aleatoriamente um
elemento da LCR.

Com isso a quantidade de soluções possíveis é ampliada porém somente soluções
promissoras são geradas.

As soluções geradas pela fase de construção do GRASP normalmente não são
localmente ótimas com relação à definição de vizinhança adotada. Surge então a
necessidade de complementar o método com a adição de uma busca local, que tem
o objetivo de melhorar a solução construída na fase de construção. O Algoritmo
\ref{alg:grasplocal} descreve um procedimento básico de busca local relativo a
uma vizinhança $N(.)$ de $s$ para um problema de minimização. A qualidade da
construção gerada causa um impacto direto na busca local, uma vez que essa
solução inicial podem constituir pontos de partidas promissores para a busca
local, permitindo assim agilizá-los.
 
\begin{figure}[h]
\caption{Pseudocódigo do procedimento de busca local do GRASP. \newline
\mbox{Fonte:
\cite{resende1995}}}\label{alg:grasplocal}
\begin{programma}
\ALGORITHM{BuscaLocal($f(.), N(.), s$)}
\STATE $V \GETS \{s{'} \in N(s) \mid f(s{'}) < f(s)\}$;
\WHILE{$\mid V \mid > 0$}
\STATE Selecione $s{'}$ de $V$;
\STATE $s \GETS s{'}$;
\STATE $V \GETS \{s{'} \in N(s) \mid f(s{'}) < f(s)\}$;
\ENDWHILE
\STATE\RETURN $s$;
\ENDALGORITHM
\end{programma}
\end{figure}

O algoritmo de busca local define no passo 1 e 5 o conjunto de vizinhos da
solução $s{'}$ que melhoram o valor de sua função objetivo. Do passo 2 à 6 a
solução corrente é atualizada enquanto houver uma solução melhor na vizinhança.

O GRASP apresenta basicamente o parâmetro $\alpha$ que pode ser ajustado.
Valores de $\alpha$ que levem a uma LCR com tamanho bastante limitado implicam
soluções próximas as da solução gulosa, obtidas com um baixo esforço
computacional, provocamdo assim uma baixa variedade de soluções construídas, que
normalmente não é interessante para a busca local já que as soluções geradas
são muito próximas. Por outro lado a escolha de valores de $\alpha$ muito
elevado implica na geração de uma grande diversidade de soluções mas, por
outro lado, muitas das soluções construídas são de baixa qualidade.

Procedimentos GRASP mais sofisticados levam em consideração a mudança do valor
de $\alpha$ ao longo das iterações de acordo com os resultados obtidos em
iterações anteriores. Estudos feitos em \cite{prais2000} indicam que essa
adaptação do valor de $\alpha$ produz soluções melhores do que aquelas obtidas
considerando-o fixo.

\subsection{ILS}

Essa seção descreve a metaheurística ILS (Iterated Local Search - Busca Local
Iterativa) que se baseia na idéia de que um procedimento de busca local
consegue melhores resultados a medida que a solução base é variada.
Esses locais diferentes são obtidos a partir de pertubações em cima da solução
ótima local corrente.

O Algoritmo \ref{alg:ils} ilustra o pseudo-código do ILS. Nele pode-se perceber
a necessidade da definição de quatro procedimentos: (a) $GeraSoluçãoInicial()$
que obtém o ponto de partida $s_{0}$ para o problema; $BuscaLocal(s)$, que
retorna o mínimo local da solução $s$, tendo como base as estruturas de
vizinhança definidas; (c) $Pertubação(histórico, s)$, que altera a solução $s$
para outra solução, e se utiliza do histórico para evitar repetir soluções bem
como para inferir o grau de pertubação necessário para escapar do mínimo local;
E o (d) $CritérioDeAceitação(s, s{''}, histórico)$, que decide em qual solução
a próxima pertubação será aplicada.

\begin{figure}[h]
\caption{Pseudocódigo do procedimento Iterated Local Search. \newline
\mbox{Fonte:
\cite{notasmarcone}}}\label{alg:ils}
\begin{programma}
\ALGORITHM{ILS}
\STATE $s_{0}$ \GETS GeraSoluçãoInicial;
\STATE $s$ \GETS BuscaLocal($s_{0}$);
\WHILE{os critérios de parada não estiverem satisfeito}
\STATE $s{'}$ \GETS Pertubação($histórico, s$);
\STATE $s{''}$ \GETS BuscaLocal($s{'}$);
\STATE $s$ \GETS CritérioAceitação($s, s{''}, histórico$);
\ENDWHILE
\STATE\RETURN $s$;
\ENDALGORITHM
\end{programma}
\end{figure}

O ILS é dependente da escolha do método de busca local, das pertubações e do critério de aceitação. Normalmente um método de descida é utilizado, mas também é possível aplicar algoritmos mais sofisticados como Busca Tabu ou outras metaheurísticas.

A intensidade da perturbação deve ser forte o suficiente para permitir escapar do ótimo
local corrente e permitir explorar diferentes regiões. Ao mesmo tempo, ela precisa ser fraca
o suficiente para guardar características do ótimo local corrente \cite{notasmarcone}.

Um aspecto importante do critério de aceitação e da pertubação é que eles induzem aos procedimentos de intensificação e diversificação. A intensificação consiste em procurar melhores soluções nas área de busca corrente, isso acontece reduzindo a força da pertubação que faz com que as novas soluções de partida se encontrem nas proximidades da anterior. A diversificação acontece com a aplicação de grandes pertubações.

\begin{figure}[ht]
	\caption{Representação esquemática do funcionamento do ILS. \newline
	\mbox{Fonte:
	\cite{notasmarcone}}}
	\label{img:ilsfuncionamento}
	\includegraphics[scale=0.3]{./img/ilsfuncionamento.png}
\end{figure}

A Figura \ref{img:ilsfuncionamento} demonstra o funcionamento do método ILS em um problema de minimização. Dado um ótimo local $s$, é realizada uma pertubação que lhe direciona para $s{'}$. Depois da aplicação da busca local, o novo mínimo $s{''}$, melhor que a anterior, é encontrada. Ou seja $f(s{''}) < f(s)$.

Uma exemplo de pertubação seria a aplicação sucessiva de estruturas de vizinhança a solução corrente.

\section{Programação Linear}

A programação linear é provavelmente a mais conhecida e utilizada técnica de
otimização em todo o mundo e geralmente é utilizada para tomada de
decisões gerenciais sobre a alocação de recursos para produção. Os custos dos recursos e
as receitas geradas pelos produtos são usados para determinar a melhor solução.
Qualquer problema que possa ser formulado com variáveis de decisão reais, tendo
uma função objetivo linear, e funções de restrição lineares, em princípio pode
ser solucionado através da programação linear. Tais programas originariamente
utilizavam o método \textit{Simplex}, porém, recentemente, métodos de
"\textit{pontos interiores}" se mostraram mais eficientes.

Embora a programação linear seja muito eficiente para a resolução de problemas
lineares, sua aplicação a problemas que apresentem objetivos ou restrições
não-lineares tem levado a problemas e falhas de modelagem. Em alguns casos,
funções não-lineares podem ser aproximadas por algumas funções lineares
conjugadas, e a programação linear ainda pode ser utilizada. Contudo, isso leva
a uma representação ineficiente do problema, podendo causar matrizes de decisão
explosivamente grandes que demandam um tempo excessivo para resolução. Esta é
uma dificuldade comum em problemas que envolvem, por exemplo,
"\textit{scheduling}" e "\textit{sequenciamento}" de processos como é o caso do
PCTA.

De forma equivalente, outros tipos de variáveis não podem ser tratadas
diretamente com o uso de programação linear. Programação inteira usa
programação linear para resolver problemas sobre variáveis inteiras, mas ainda
com funções objetivo e restrições puramente lineares. As variáveis inteiras são
representadas como variáveis reais no algoritmo de resolução do problema. Então
um processo repetitivo é usado para "delimitar" o valor destas variáveis em
valores inteiros, através da adição de restrições e reprocessamento da solução.
Esse método, conhecido como \textit{"branch \& bound"}, finaliza quando todas
as variáveis assumem valores inteiros. Quando o número de variáveis inteiras é
pequeno, a programação inteira soluciona o problema rapidamente. Infelizmente
esse procedimento pode consumir muito tempo com um número grande de variáveis
inteiras, podendo, em alguns casos, necessitar de milhões de iterações para
serem resolvidos.
 	
Essa técnica foi muito utilizada na segunda guerra mundial para otimizar as
perdas inimigas e reduzir o custo das operações e também é utilizado no
planejamento de algumas empresas.



\chapter{Revisão da literatura}
  
	
O trabalho de Argüello e Bard \cite{arguelo1997}  apresenta um
algoritmo baseado no GRASP para resconstruir trilhos de aeronaves que tenham
sofrido atrasos durante o decorrer do dia e tem como principal objetivo a
reducão dos custos da reatribuição das aeronaves aos voos que é mensurado
apartir do atraso dado aos voos e pelo número de voos cancelados. 

Nesse trabalho foi utilizado a ideia de trilhos cancelado, que é formado
pelos voos que não serão levados em consideração na solução final. 
		
A reconstrução dos trilhos é feita com a utilização sucessiva de três
estruturas de vizinhança, \textit{flight routing augmentation}, \textit{partial route exchange} e
\textit{simple circuit cancelation} onde as duas primeiras são aplicadas em um
par de trilhos e a terceira é aplicada em trilhos individualmente

O \textit{flight routing augmentation} remove uma sequência de voos do
primeiro trilho e acrescenta eles no trilho de destino, ou seja, o segundo
trilho é acrescido dos voos que foram removidos do primeiro. O trilho de
destino pode crescer de três formas. Primeiro um circuito pode ser inserida
no seu início. Um circuito é uma sequência de voos que se origina e termina no
mesmo aeroporto. A segunda forma é a adição de um circuito em algum lugar
entre o primeiro e o ultimo voo. A terceira forma envolve a adição de uma
sequência de voos, que não precisa ter a mesma origem e destino, e a sua
inserção no final do segundo trilho. Lembrando que apenas movimentos viáveis
são avaliados. 
		
O movimento de \textit{partial route exchange} é uma simples
troca de um par de sequências de voos. Dois tipos de trocas são possíveis. A
primeira é a troca de duas sequências que possuam os mesmo extremos. E a
segunda é uma troca que resulta na mudança do aeroporto de destino. Um trilho
de cancelamento não pode trocar seus aeroportos de destino com outro trilho
pois esse movimento poderia causar uma violação na restrição de balanceamento
de aeronáves.
		
O \textit{simple circuit cancelation} é feito em um único trilho e ela
simplesmente remove um circuito desse trilho e efetua a criação de um novo
trilho de cancelamento. Além disso foi desenvolvido um modelo matemático que
foi utilizado apenas para a obtenção de um limite inferior (\textit{lower
bound}).
		
		
Mercier e Soumis \cite{mercier2007} resolveram o PCTA em conjunto com
o problema de escala de tripulantes pois Cordeau et al. \cite{cordeau2001},
Klabjan et al. \cite{klabjan2002} e Cohn e Barnhart \cite{mainville2003}
mostraram que a resolução desses problema de forma integrada pode gerar
soluções que são significantemente melhor que as geradas de forma sequencial.
Com essa finalidade eles proporam uma formulação compacta do problema e
utilizaram o método de decomposição de Benders com um procedimento de geração
de restrição dinâmica para resolve-lo. Com a agregação desses dois problemas a
resolução se tornou pesada e viável apenas para instâncias diárias. Os testes
do algoritmo foram baseados em instâncias contendo no máximo 500 voos que
foram fornecidas por duas grandes companhias aereas, porém elas não se
encontram disponíveis no artigo. 
		
Pontes R., et al \cite{pontes2002} utilizaram a fase de construção do GRASP
para resolver o PCTA, também propuseram um modelo matemático que foi
adaptado para auxiliar na geração da nossa solução. Além disso uma instância
da Rio-Sul foi disponibilizada para a realização de testes. Com o solver eles
conseguiram obter a solução ótima dessa instância mas o autor informou que
essa resolução demorou dias para finalizar. Com a utilização da heurística
eles conseguiram apenas se aproximar dessa solução porém com um tempo de 384
segundos.
		
Em \cite{mohamed2011} Mohamed et al. resolveu de forma integrada o problema
de atribuição de frota e o problema de construção de trilhos de aeronaves,
para uma pequena empresa de aviação a TunisAir. Além disso as restrições de
manutenção não foram levadas em consideração pelo fato dela poder ser feita
em todos os aeroportos em que as aeronaves passam a noite.
		
%GRASPs have been used to find high quality solutions to a variety of logistics and combi- natorial optimization problems including maintenance base %planning (Feo and Bard, 1989), machine scheduling (Feo et al., 1991), and number partitioning (Argu ̈ello et al., 1996) to name a few.
%ão
 
\chapter{Descrição do problema}\label{cap:descprob}

%%%Aqui parece ser descrição do problema
  Os voos são organizados de forma que cada aeronave seja responsável por uma sequência válida, que é chamada de trilho.  O conjunto desses trilhos é a solução do problema e é denominada de malha. Essa malha deve conter o menor número possível de trilhos que atenda todos os voos planejados com o mínimo de modificações.
  
  As restrições que envolvem o PCTA induzem a formação de uma rede de possíveis conexões. Nessa rede os nós representam os voos e os arcos representam as conexões possíveis entre esses voos. Dessa forma o problema pode ser formulado como um problema de minimização de custos em uma rede.
  
  Dado a possibilidade de mudanças no tempo de partida sugerido dos voos e também a permissão para criar voos de
reposicionamento, uma grande quantidade de soluções podem ser geradas. A ligação dos voos pode ocorrer de 6 formas
distintas aqui denominado tipos de arcos. Os arcos do tipo 1 permitem a ligação de voos sem a utilização de atrasos e/ou
reposicionamento. Os arcos do tipo 2 utilizam atrasos mas não o reposicionamento. Os arcos do tipo 3 permitem o
sequenciamento com a utilização de um voo de reposicionamento mas sem inserir atraso em nenhum dos voos envolvidos. Os
arcos do tipo 4 utilizam-se de atrasos e de um voo de reposicionamento para fazer a ligação entre dois voos. Os arcos do tipo 5
e 6 são usados no modelo, que é baseado no fluxo em grafo, para representar respectivamente o nó origem(source) e o
destino(sink). Abaixo um maior detalhamento desses arcos:

  
  \begin{itemize}
\item Conexão natural (Arco do tipo 1) - Os arcos desse tipo conectam dois voos respeitando o tempo de partida sugerido e
a restrição geográfica. Eles são associados com as ligações que não requerem mudanças no tempo de partida e nem
       precisam de voos de reposicionamento. O arco do tipo 1 não apresenta custo para ser adicionado a solução.

\item Mudança no tempo (Arco do tipo 2) - Apesar de ter os voos incidentes no mesmo aeroporto, os arcos desse tipo não
permitem a ligação de forma direta pois o tempo de solo disponível não é suficiente para permitir a ligação. No
      entanto, a escolha desse tipo de arco implica em uma mudança no tempo de partida sugerido para quaisquer um dos
      voos envolvidos. O custo de um arco desse tipo é igual a soma dos valores absolutos dos atrasos (em minutos) dos
      horários de partida envolvidos.

\item Voos de reposicionamento (Arco do tipo 3) - Esses arcos representam conexões entre dois voos em que a origem
parte de um aeroporto diferente do local de partida do voo de destino, no entanto, existe tempo suficiente para um voo
de reposicionamento, entre os dois locais, sem violar as restrições de tempo de solo. Os custos de um arco do tipo 3 é
  igual a duração do voo de reposicionamento, incluindo o seu tempo de solo.

\item Voos de reposicionamento mais mudança de tempo (Arco do tipo 4) - Esses arcos representam conexões que
precisam de um voo de reposicionamento mais mudança no tempo de partida sugerido. O arco do tipo 4 tem custo
igual tempo do voo de reposicionamento, incluindo o tempo de solo, mais a soma dos atrasos dos horários de partida
envolvidos em valor absoluto.

\item Arcos do nó fonte ou \textit{source} (Arco do tipo 5) - Esses, arcos são criados para identificar o inicio de um trilho e é com
 ele também que se sabe a quantidade de trilhos necessários para resolver o problema. Cada arco do tipo 5 tem o custo
  igual a 1000.

\item Arcos do nó final ou \textit{sink} (Arco do tipo 6) - Esses arcos tem como destino o nó fictício que é usado para finalizar um
  trilho no modelo. Os arcos do tipo 6 não tem custo.


\end{itemize}
  
%%%% Até aqui %%%%%%%%%%

\chapter{Método Proposto} \label{cap:metodoprop}
  
O método proposto se utiliza do GRASP, do ILS e da abordagem exata através da
programação linear inteira, pretendendo tirar proveito das vantagens de cada
uma dessas técnicas. Ou seja, combinando a agilidade dos métodos heurísticos com
a optimalidade do método exato.
  
Da mesma forma que em outras abordagens heurísticas, esse novo algoritmo
consome pouco tempo computacional, em relação ao método exato e tem a capacidade
de escapar de mínimos locais.
  
O GRASP foi utilizado como a base do algoritmo, onde a parte da construção
seguiu a sua definição padrão, com a geração de uma lista restrita de candidatos
(LRC) e a posterior escolha aleatória entre esses elementos. A parte da busca
local foi adaptada para executar em conjunto com o ILS modificado. Para o ILS
foram definidos algumas estruturas de vizinhança que foram utilizadas
na busca local, e a perturbação foi feita com a utilização de um \textit{solver}
em uma parte do problema. Essa abordagem permite que o algoritmo gere boas
soluções e escape de mínimos locais além de promover uma aceleração na obtenção
de boas soluções, pois quando o \textit{solver} encontra uma melhor solução ele
consegue mudar o espaço de soluções em que a busca era efetuada.


O \textit{solver} é utilizado para resolver um modelo matemático que foi
desenvolvido baseado na proposta de \cite{pontes2002} que é aplicado a uma parte do problema cada
vez que se deseja fazer uma perturbação. Enquanto a busca local usa o método de
descida, variando entre três estruturas de vizinhança, o \textit{swap-x}, o
\textit{crossover} e a \textit{compactação}. Mais adiante serão dado mais
detalhes sobre o modelo matemático, a forma de escolha do sub problema, da fase
de construção que foi implementada, da busca local e das implementações que não
tiveram êxito.

\section{Modelo matemático} \label{sec:modelomat}

   
A modelagem proposta por \cite{pontes2002} aborda todas as restrições do
problema fazendo com que a quantidade de restrições geradas seja muito elevada.
A idéia utilizada na nossa formulação é a de tentar reduzir ao máximo a
quantidade de restrições necessárias. Isso é feito com a modelagem de apenas
algumas restrições, aquelas que são possíveis no mundo real.

Primeiro se percebeu que não há necessidade de modelar os 4 tipos de arcos para
cada voo, uma vez que dados dois voos só pode vir a ocorrer dois tipos de arcos
possíveis entre eles. Essa situação é ilustrada na Figura
\ref{fig:modelagem_arcos}.

\begin{figure}[ht]
	\centering
	\caption{Arcos necessários para ligar dois voos. \newline \mbox{Fonte:
	(Própria)}}\label{fig:modelagem_arcos}
	\includegraphics[scale=0.4]{./img/modelagem_arcos}
\end{figure}


Em a) os voos respeitam a restrição geográfica, dessa forma apenas os arcos de
tipo 1 e 2 precisam ser modelados uma vez que não teria sentido fazer um voo de
reposicionamento nessa situação. Em b) os aeroportos em questão são diferentes,
sendo necessário apenas a modelagem dos arcos do tipo 3 e 4, perceba que não
teria outra alternativa se não fazer um voo de reposicionamento.

\subsection{Definição}

Seja $D = (V,A)$ um grafo representando uma instância do PCTA, onde o conjunto
de vértice $V = {v_{i}:i \in I}$ de D é indexado por $I = {1, 2, ..., n+1,
n+2}$ onde $v_{n+1}$ e $v_{n+2}$, identificam, respectivamente, os nós fonte e
destino. E os nós restantes referem-se ao conjunto de nós originais, com $n$
elementos. Sejam os custos ${c_{ij}:(i,j) \in A}$ introduzidos acima, estando
associados com cada arco do grafo.
  
Seja ${x_{ij}:(i,j) \in A}$ um conjunto binário 0-1 de variáveis usada para
controlar a inclusão $(x_{ij} = 1)$ ou a exclusão $(x_{ij} = 0)$ de um arco
(possível conexão) entre vértices (voos) $v_{i}$ e $v_{j}$. O conjunto
$\overline{I}$ identifica o conjunto de nós excluindo o nó fonte $(v_{n+1})$ e
o nó de destino $(v_{n+2})$. A função objetivo foi dividida para facilitar o
entendimento de como é feito o custo de adicionar um trilho. Variáveis reais
$\delta_{i}$ e $\theta_{i}$, $i \in \overline{I}$ são usados para representar,
respectivamente, o desvio do tempo de partida sugerido e a norma desse desvio
para $v_{i}$. Essas variáveis devem no entanto obedecer $-\gamma_{i} \geq
\delta_{i} \geq \gamma_{i}$ e $0 \geq \theta_{i} \geq \gamma_{i}$, onde
$\gamma_{i}$ é o valor máximo de desvio permitido (em cada direção) do tempo de
partida sugerido para o voo. Finalmente o tempo de partida sugerido que é dado
por $s_{i}:i \in \overline{I}$.
  
\section{Função objetivo}

\begin{equation}
Minimizar \  \ \sum_{j \in \overline{I}} x_{v_{n+1}j}(CUSTO\_TRILHO) + \sum_{i \in
\overline{I}} \sum_{j \in I} x_{ij}c_{ij} + \sum_{i \in
\overline{I}} \theta_{i}
\end{equation}

\section{Restrições}

\begin{enumerate}


\item[a)] Garantia de recobrimento dos voos \\
\begin{equation}
  \sum_{i \in I} x_{ij}= 1 \   \ \forall_{j} \in \overline{I} 
\end{equation}
\begin{equation}
\sum_{j \in I} x_{ij} = 1 \   \ \forall_{i} \in \overline{I}
\end{equation}


\item[b)] Viabilidade das conexões \\
\begin{equation}
s_{i} + t_{i}x_{ij} - M(1 - x_{ij}) + \delta_{i} \leq s_{j} + \delta_{j} \   \ \forall_{i,j} \in \overline{I}
\end{equation}
%\begin{equation}
%\sum_{i \in I} x_{i(n+1)} = 0
%\end{equation}
%\begin{equation}
%\sum_{j \in I} x_{(n+2)j} = 0
%\end{equation}

\item[c)] Modulo do desvio do tempo de partida sugerido \\
\begin{equation}
\theta_{i} \geq \delta_{i} \   \ \forall_{i} \in \overline{I}
\end{equation}
\begin{equation}
\theta_{i} \geq -\delta_{i} \   \ \forall_{i} \in \overline{I}
\end{equation}

\item[d)] Limites das variáveis \\
\begin{equation}
0 \geq \theta_{i} \geq \gamma_{i} \   \ \forall{i} \in \overline{I}
\end{equation}
\begin{equation}
-\gamma_{i} \geq \delta_{i} \geq \gamma_{i} \   \ \forall_{i} \in \overline{I}
\end{equation}
\begin{equation}
x_{ij} \in \{0,1\}
\end{equation}
\end{enumerate}

\clearpage

A função 1 do modelo representa a função objetivo onde a primeira parte
representa o custo de inicializar um trilho, ou seja, sempre que um arco
partindo de $n+1$, que é o nó origem, for ativado acrescenta-se o custo de um
trilho a solução. A segunda parte representa o custo de adicionar os demais
arcos e a terceira parte representa a adição do módulo dos atrasos que foram
utilizados. Perceba que a primeira e a segunda parte podem ser modeladas juntas,
a sua separação foi feita para facilitar o entendimento do modelo.

As restrições 2 e 3 representam respectivamente que todos os nós deverão
ter apenas um antecessor e um sucessor, as únicas exceções são o nó origem que
pode incidir em vários outros nós e o nó de destino que pode ser incidido por
vários nós.

A restrição 4 é utilizada para manter a viabilidade das conexões acrescentando
atrasos a elas se for necessário. Se $x_{ij} = 1$ então tem-se que o tempo de
partida do voo i ($s_{i}$) mais a duração do voo do seu voo ($t_{i}$) mais um
possível atraso nesse voo ($\delta_{i}$) deve ser menor ou igual ao tempo de
partida do voo j ($s{j}$) mais um atraso que lhe seja dado ($\delta_{j}$). Se
$x_{ij} = 0$ então M irá validar automáticamente essa restrição. M tem valor
igual a soma da duração dos voos. As restrições 5, 6 e 7 modelam o valor
absoluto do atraso em $\theta$.

A equação 8 restrige o atraso/adiantamento máximo que pode ser utilizado e a
equação 9 modela a variável de decisão $x$ como sendo booleana.

Pode-se perceber que o modelo matemático não faz menção ao tempo de solo ($g$).
Isso ocorre pois esse tempo é incorporado ao voo como demonstrado na Figura
\ref{fig:conversion}, ou seja o tempo de partida sugerido $s$ passa a ter o
valor $s - g$ e a duração $t$ do voo passa a ter o valor $t + g$. Uma vantagem
de usar essa abordagem que integra o tempo de solo ao voo é a redução da
quantidade de restrições do problema. 



\begin{figure}[ht]
	\caption{Conversão de um voo para ser utilizado no
	solver. \newline \mbox{Fonte: (Própria)}}\label{fig:conversion}
	\includegraphics[scale=0.4]{./img/conversion}
	
\end{figure}

Além disso o conjunto $A$ contém apenas um tipo de arco, o arco do tipo 1 se os
voos satisfazem a restrição geográfica e o arco do tipo 3 caso não satisfaçam.
Os arcos dos tipos 2 e 4 são modelados a partir  da variável $\delta$ que tem
seu custo acrescentado na função objetivo.

Essa estratégia permite a redução de 3 arcos para cada voo, o que deixa o
modelo mais leve.

O calculo dos custos são feitos através de um pré-processamento, onde os arcos
viáveis recebem os valores referentes ao seu tipo, por exemplo, no caso de um
arco originário do nó origem, arco do tipo 6, um custo 1000 é atribuído. No
caso de arcos que deverão ser evitados um custo elevado é atribuído.
  	
  	
%\section{Pré-processmanto da instância}

%No caso da instância possuir mais de um dia de operação então pode-se quebra-la
%em dias se houver tempo vago entre os dias dessa instância.
  
\section{Fase de construção do GRASP}\label{sessao:construcao}
  
A construção da solução é feita elemento a elemento utilizando o
GRASP. Inicialmente é feita a ordenação do conjunto de voos a partir do seu
tempo de partida sugerido. O algoritmo só termina quando todos os voos já
foram alocados em algum trilho.
  
Existem duas formas de fazer a montagem da solução, uma seria a montagem de
trilhos de forma sequencial, onde um novo trilho só é iniciado quando o anterior
se encontra saturado. A outra forma é a montagem de trilhos de forma paralela,
que, a priori, provocaria uma melhor distribuição dos voos. Na prática a
primeira abordagem foi adotada, pois, nas instâncias disponíveis ela apresentou,
sempre, soluções de melhor qualidade. 

Pode ser que para instâncias com alguma característica específica a
estratégia de montagem dos trilhos de forma paralela pode apresentar melhores
resultados.


  
\subsection{Formação dos trilhos de forma sequencial}

Quando se pensa na escolha do primeiro voo do trilho, a decisão imediata é a
escolha do voo que contenha o menor horário de partida sugerido, ou seja, o voo
mais próximo. Porém essa escolha reduz a quantidade de soluções que podem ser
geradas, isso ocorre pois o primeiro voo tem uma grande influência nas
possíveis soluções que um trilho pode assumir. Para evitar isso a escolha do
primeiro voo de um trilho é feita baseando-se nos 5 voos com menor horário de
partida que ainda não estejam alocados a nenhum outro trilho. 

Esses voos são adicionados a lista de candidatos iniciais (LCI) em seguida é
feita a escolha do elemento que irá iniciar o novo trilho levando em
consideração apenas os elementos que possuam o horário de partida distante de
até $\alpha \%$ do voo de menor horário de partida, esses elementos são então
inseridos em uma lista restrita inicial (LCI) na qual é feita a seleção
aleatório de um dos voos. Isso é feito para evitar a escolha de um voo muito
distante do menor voo.

Os pseudocódigos \ref{alg:selectinit} e \ref{alg:formseq} ajudam a elucidar esse
entendimento. Primeiro deve-se entender o pseudocódigo \ref{alg:selectinit}
onde é feito a escolha de um voo que deverá iniciar um trilho, esse
pseudocódigo serve tando para montar trilhos de forma sequencial como de forma
paralela. Em 1 é formada a LCI que é composta pelos 5 primeiros voos ainda não
alocados, de 2 a 4 é feita a formação da LRI tendo como base o menor horário de
partida $h(v_{min})$, o maior horário de partida $h(v_{max})$ e o $\alpha$ e em
5 é feita a escolha aleatória de um desses voos. Já no pseudocódigo
\ref{alg:formseq} é feita a montagem dos trilhos de forma sequencial. Em 1 os
voos disponíveis são ordenados de acordo com seu horário de partida, em 2 a
malha, que será o resultado, é inicializada com vazio. De 3 a 9 é feita a
montagem dos trilhos que irão compor a malha, em 4 e 5 é obtido o primeiro voo
do trilho, em 6 é feita a montagem completa desse trilho, em 7 os voos
pertencente ao trilho corrente são removidos do conjunto de voos disponíveis e
em 8 o trilho corrente é adicionado a malha resultante.

 \begin{figure}[h]
\caption{Pseudocódigo do procedimento de seleção de um voo inicial. \newline
\mbox{Fonte: Própria}}\label{alg:selectinit}
\begin{programma}
\ALGORITHM{selecionaVooInicial(V)}

\STATE $LCI$ \GETS cincoPrimeirosVoos($V$);
\STATE $h(v_{min}) \GETS min\{h(v) \mid v \in LCI\}$;
\STATE $h(v_{max}) \GETS max\{h(v) \mid v \in LCI\}$;
\STATE $LRI \GETS \{v \in LCI \mid h(v) \leq h(v_{min}) + \alpha(h(v_{max}) -
h(v_{min}))\}$;
\STATE Selecione aleatoriamente um elemento $v \in LRI$;
\STATE\RETURN $v$;
\ENDALGORITHM
\end{programma}
\end{figure}

\begin{figure}[h]
\caption{Pseudocódigo do procedimento de formação sequencial dos trilhos.
\newline
\mbox{Fonte: Própria}}\label{alg:formseq}
\begin{programma}
\ALGORITHM{formaçãoSequencialDosTrilhos(V)}

\STATE Ordene o conjunto de voos não alocados $V$;
\STATE $M \GETS \emptyset$;
\WHILE{$V \neq \emptyset$}
\STATE $v \GETS selecionaVooInicial(V)$
\STATE $T \GETS \{v\}$;
\STATE $T \GETS completaTrilho(T)$;
\STATE $V \GETS V - \{v \in T\}$;
\STATE $M \GETS M \cup T$;
\ENDWHILE
\STATE\RETURN $M$;

\ENDALGORITHM
\end{programma}
\end{figure}


\subsection{Formação dos trilhos de forma paralela}
 
Nessa estratégia um trilho é iniciado sempre que existe um voo que não pode ser
inserido em nenhum dos trilhos que estejam sendo montados, mantendo-se assim um
conjunto de trilhos disponíveis (CTD).

Em cada iteração o trilho corrente (TC) é escolhido a partir do CTD de forma
aleatória. Feito isso, adiciona-se um voo a esse trilho. Caso não existam voos
candidatos para adição no TC este é removido da CTD e uma nova iteração é
iniciada.

O pseudocódigo \ref{alg:formparalel} ajudam no entendimento do algoritmo. Em 1
os voos disponíveis são ordenados de acordo com seu horário de partida, em 2 e 3
respectivamente a malha resultante e o CDT são inicializados com vazio. De 4 a
19 é feita a montagem paralela dos trilhos que irão formar a malha. Em 5 é feita
a seleção de um voo inicial. Em 6 verifica-se se existe algum trilho corrente
que pode incorporar esse voo, se existir então em 7 é feita a seleção de um
trilho corrente (TC) a partir dos elementos da CDT e em 8 seleciona-se um voo
que possa ser inserido em TC e insere o voo em 13. 	Se não existir nenhum
voo que possa ser inserido em TC então este é removido da CDT como pode ser
visto de 9 a 12. 

De 15 a 18 é criado um novo trilho sempre que não existe nenhum trilho da CDT
que possa comportar o voo selecionado em 5. Em 20 se insere os trilhos
pertencentes a CDT na malha resultante. Isso ocorre porque todos os voos já
foram alocados em algum trilho.



\begin{figure}[h]
\caption{Pseudocódigo do procedimento de formação paralela dos trilhos.
\newline
\mbox{Fonte: Própria}}\label{alg:formparalel}
\begin{programma}
\ALGORITHM{formaçãoParalelaDosTrilhos(V)}

\STATE Ordene o conjunto de voos não alocados $V$;
\STATE $M \GETS \emptyset$;
\STATE $CTD \GETS \emptyset$;
\WHILE{$V \neq \emptyset$}
\STATE $v \GETS selecionaVooInicial(V)$
\IF {$v$ pode ser inserido em um trilho do CTD}
	\STATE $TC \GETS escolheTrilhoAleatório(CTD)$;
	\STATE $nv \GETS selecionaVooCandidato(TC)$;
	\IF {$nv = \emptyset$}
		\STATE $CDT \GETS CDT - \{TC\}$
		\STATE $M \GETS M \cup \{TC\}$
	\ELSE
		\STATE $TC \GETS TC \cup \{nv\}$
	\ENDIF
\ELSE
	\STATE $T \GETS \{v\}$; 
	\STATE $CTD \GETS CTD \cup T$ 	
\ENDIF
\ENDWHILE

\STATE $M = M \cup \{t \in CTD\}$;
\STATE\RETURN $M$;

\ENDALGORITHM
\end{programma}
\end{figure}


  
\subsection{Escolha dos voos de um trilho}

A escolha do primeiro voo de um trilho é feita como explicado nas seções
anteriores enquanto os demais voos são escolhidos tendo como base um tipo de
arco e uma lista restrita de candidatos (LRC).
 
Os tipos de arcos foram definidos no Capítulo \ref{cap:descprob}, porém nessa
etapa apenas 4 tipos são considerados, o   $a_{1},a_{2},a_{3},a_{4}$ que
representam formas de ligações entre os voos. Os arcos do tipo 5 e 6 só são
utilizados na modelagem matemática. Os arcos do tipo 1 permitem a
ligação de voos sem a utilização de atrasos e/ou reposicionamentos. Os arcos do
tipo 2 utilizam atrasos mas não o reposicionamento. Os arcos do tipo 3 permitem
o sequenciamento com a utilização de um voo de reposicionamento mas sem inserir
atraso em nenhum dos voos envolvidos. Os arcos do tipo 4 utilizam-se de atrasos
e de um voo de reposicionamento para fazer a ligação entre dois voos. Os arcos
do tipo 5 partem do nó \textit{source} e servem para modelar o início de um
trilho. Os arcos do tipo 6 representam o no de destino (\textit{sink}), nele
incidem todos os últimos voos de cada trilho.

Primeiramente é feita a escolha do tipo de arco que será utilizado para efetuar
a ligação do ultimo voo do trilho corrente. Essa escolha é feita tendo
como base as probabilidades de cada um desses arcos acontecer. Essa
probabilidade foi definida como sendo
$P(a_{1})=0.79,P(a_{2})=0.16,P(a_{3})=0.04,P(a_{4})=0.01$ pois a solução ótima
do problema real da Rio Sul apresentava essas características. Esses valores
são empíricos e se alterados podem vir a melhorar, ou piorar, a qualidade da
solução, dependendo das características da instância.

De posse do tipo de arco, é feita então a formação da lista de candidatos. Essa
lista é ordenada de acordo com o seu horário de partida sugerido, caso o arco
seja do tipo $A_{1}$, ou pelo custo associado a sua escolha para os demais
tipos de arco. No caso da lista de candidatos não possuir nenhum voo, então
outro tipo de arco é sorteado, até que não seja possível acrescentar voos ao
trilho de nenhuma forma, quando isso ocorrer a construção
desse trilho é finalizada.
 
Caso seja possível a obtenção de uma lista de candidatos então ela é reduzida
tendo como base o passo 2 a 4 do algoritmo \refa{alg:selectinit} formando
assim a lista de candidatos restrita (LCR), essa redução remove os candidatos que estão
muito afastado do melhor candidato da lista. Como está lista se encontra
ordenada, então, o elemento de menor impacto ($v_{menor}$) na solução é o
primeiro e o de maior impacto ($v_{maior}$) é o último. A LCR contém os
candidatos que tenha o valor de impacto na solução entre $v_{menor}$
e $v_{menor} + \alpha*(v_{maior} + v_{menor})$, onde $\alpha$ é o
grau de gulosidade do GRASP, que nesse trabalho foi definido como sendo 0.5. O
candidato deve ser escolhido de forma aleatória entre os elementos da LCR.

O pseudocódigo \ref{alg:calcvoo} explica como isso acontece. Em 1 inicializa o
conjunto de arcos com os 4 tipos possíveis em 2 realiza-se um sorteio do tipo de
arco que irá ser inicialmente avaliado levando em consideração a probabilidade
de cada um deles acontecer ($P(a_{1}), P(a_{2}), P(a_{3}), P(a_{4})$). Em 3 é
feita a remoção do tipo de arco selecionado do conjunto inicial e em 4
é selecionado o voo corrente que necessita de um sucessor, nesse caso o último
voo do trilho.

De 5 a 13 procura-se um candidato para o voo corrente levando em consideração
todos os tipos de arcos, iniciando pelo que foi selecionado no passo 2.

\begin{figure}[h]
\caption{Pseudocódigo de calculo do proximo voo de um trilho
\newline
\mbox{Fonte: Própria}}\label{alg:calcvoo}
\begin{programma}
\ALGORITHM{obtemProximoVoo(T,V)}

\STATE $A \GETS \{1,2,3,4\}$;
\STATE $a \GETS sorteaTipoDeArco(A, P(a_{1}),P(a_{2}),P(a_{3}),P(a_{4}))$;
\STATE $A \GETS A - \{a\}$; \STATE $v \GETS ultimoVoo(T)$;

\FOR{$i$ \FROM $1$ \TO $4$}\PGlnlabel{forline}
\STATE $c \GETS proximoCandidato(v, V, a)$;
\IF {$c = \emptyset$}
	\STATE $a \GETS proximoArco(A)$;
	\STATE $A \GETS A - \{a\}$;
\ELSE
	\STATE\RETURN $c$;
\ENDIF 
\ENDFOR

\STATE\RETURN $\emptyset$;

\ENDALGORITHM
\end{programma}
\end{figure}
 
 \section{Fase de busca local}

Com a finalização da etapa anterior tem-se uma solução do problema. A fase de busca
local efetua modificações nessa solução com a finalidade de obter outras
melhores que estejam próximos a ela, isso é feito através da aplicação das
estruturas de vizinhanças. No método proposto essa fase foi implementada através
da utilização da metaheurística ILS que alterna busca local com pertubações
conseguindo assim escapar de mínimos locais, ou seja, primeiro são aplicados as
estruturas de vizinhança, visando obter o valor ótimo local da solução, depois
é feita uma perturbação que diversifica a solução. Isso é feito através da
aplicação do modelo matemático, definido no início desse capítulo, em uma parte
do problema. Quando nenhuma das duas estratégias consegue melhorar a solução
então a busca local encerra e uma nova iteração do GRASP pode ser iniciada.
 
\subsection{Estruturas de vizinhança}
 
Foram definidas três estruturas de vizinhança para serem utilizadas na busca
local, o Swap-X e o Cross-Over, que tem o objetivo de remover modificações nos
horários de partida sugeridos dos voos, e a Compactação, que promove a redução
do número de trilhos. Abaixo essa estruturas são explicadas.
 
\subsubsection{Swap-X}

Esse operador efetua a troca de X voos de um trilho por um conjunto de voos de
outro trilho. Dessa forma pode-se conseguir remover os atrasos que foram criados
na etapa de construção. No método proposto apenas os movimentos do tipo Swap-1
e Swap-2 são utilizados, pois foram as que obtiveram mais êxito em melhorar a
solução. A Figura \ref{fig:swap1} mostra um caso em que a aplicação dessa
estrutura permite remover um atraso.

\begin{figure}[ht]
	\caption{Estrutura de vizinhança Swap-1. \newline \mbox{Fonte:
	(Própria)}}\label{fig:swap1}
	\includegraphics[scale=0.4]{./img/swap-1}
	
\end{figure}
 
 \subsubsection{Cross-Over}
 
A ideia do operador $crossover$ é a de efetuar troca entre dois segmentos de
trilhos com a finalidade de gerar novos trilhosque não apresentes voos
atrasados. A Figura \ref{fig:crossover} mostra um caso em que a aplicação dessa
estrutura melhroa a solução.


\begin{figure}[ht]
	\caption{Estrutura de vizinhança CrossOver. \newline \mbox{Fonte:
	(Própria)}}\label{fig:crossover}
	\includegraphics[scale=0.4]{./img/crossover}
	
\end{figure}
 
 \subsubsection{Compactação}
 
A compactação é a única estrutura de vizinhança utilizada que é capaz de
reduzir a quantidade de trilhos da solução final. Isso ocorre porque ela
consegue inserir um trilho em outro de forma direta ou com a utilização de um
voo de reposicionamento.
 
A figura \ref{fig:compactacao} mostra a redução de um trilho com a utilização
desse movimento.

\begin{figure}[ht] 
	\caption{Estrutura de vizinhança Compactação. \newline \mbox{Fonte:
	(Própria)}}\label{fig:compactacao}
	\includegraphics[scale=0.4]{./img/compactacao}
	
\end{figure}
 
 \subsection{Perturbação usando o método exato}
   
A perturbação é utilizada quando as estruturas de vizinhança não
conseguem melhorar a solução, quando isso ocorre, pode-se dizer que a
solução corrente é a ótima local com relação a vizinhança que foi definida.
 
Para tentar encontrar outros mínimos locais aplica-se uma modificação na
estrutura da solução, mesmo que isso provoque uma piora na sua qualidade. Isso
se mostra interessante para o algoritmo pois ele irá efetuar busca de melhorias
em outros locais no espaço de soluções através da sua busca local.
 
O método de perturbação utilizado aqui difere do que normalmente é aplicado
pois a solução tem a sua estrutura modificada e ainda consegue melhorar a sua
qualidade. Isso é feito através da resolução do modelo matemático em uma
parte do problema. A sua utilização ocorre com a seleção de um conjunto de
trilhos, e sua a posterior aplicação no \textit{solver} configurado para o
conjunto de voos da seleção.

O método exato retorna a configuração ótima desses voos, que
são agrupados novamente a solução antiga. O solver tem um tempo máximo
estabelecido e se não retornar nenhuma solução nesse tempo considera-se que não
houve melhora e uma nova iteração é iniciada.

A seleção dos trilhos é feita com base no seu
\textit{grau de compactação} que é definido como sendo porcentagem de
utilização efetiva de um trilho com relação ao tempo de partida do primeiro voo
e o tempo de chegada do último voo da instância, ou seja, quanto maior o tempo
que a aeronave, que opera um determinado trilho, permanece voando maior será o
seu grau de compactação. O calculo do grau de compactação não leva em
consideração os voos de reposicionamento, pois eles não estão no planejamento
inicial e por isso não são passados para o modelo.

Existe três formas de fazer a seleção desses. Pode-se adicionar os trilhos que
possuem o maior grau de compactação, ou adicionar os trilhos que possuem o
menor grau de compactação ou pode-se alternar entre a escolha de um trilho com
o maior grau de compactação e um de menor grau de compactação.

Na prática adotou-se a segunda abordagem, selecionando os trilhos de menor grau
de compactação, pois os resultados foram superiores aos das outras abordagens.
 
Os trilhos são adicionados a solução até o limite de 80 voos, pois o solver
conseguiu em nossos experimentos resolver um problema desse porte de forma
imediata. Isso ocorre porque em nossos experimentos o \textit{solver} conseguiu
resolver um problema desse porte de forma quase imediata.

\subsection{Método de construção paralela}

A estratégia paralela foi desenvolvida para aumentar a escalabilidade na
resolução de grandes problemas e também na tentativa de obter melhores soluções
em um menor espaço de tempo. O seu funcionamento parte da divisão da instância
em períodos de tempos iguais que são distribuídas para cada processo. Em cada
período os processos realizam uma construção do algoritmo descrito na sessão
\ref{sessao:construcao}. Ao final dessa etapa cada processo possui um um
conjunto de trilhos que devem ser ligados uns com os outros para a geração de
uma malha de voo completa.
 
O algoritmo é dividido em passos onde em cada passo um
nível diferente de vizinho é visitado na tentativa de efetuar a ligação dos seus
trilhos. No nível 1 é feita a comunicação seguindo a topologia de uma
rede em anel, onde cada processo fala com seu vizinho sucessor a uma distância
de 1, ou seja, o processo 0 fala com o processo 1, o 1 com o 2 e assim por
diante até que o último processo fale com o processo 0. Quando todos as tentativas de
ligação de trilhos entre esses processo são esgotados então o passo do algoritmo
é incrementado para que o processo consiga conversar com o processo sucessor que
possua uma distância de 2 dele, ou seja, o processo 0 se comunica com o processo
2, o 1 com o 3 e assim por diante, isso ocorre até que todos os processo tenham
se comunicados com os outros ou até que todos os trilhos estejam ligados.

Essa estratégia é interessante pois normalmente as empresas trabalham com uma
malha circular que costuma se repetir toda a semana.

O pseudocódigo \ref{alg:metodoparalel} simplifica o entendimento do algoritmo.
Perceba que a complexidade de execução e de mensagens do algoritmo distribuído
\ref{alg:metodoparalel} no pior caso é de $O(s.c)$, onde $s$ é o número de
processos e $c$ é a maior quantidade de trilhos de um processo.


\begin{figure}[h]
\caption{Pseudocódigo do método de comunicação paralela usado na construção dos
trilhos.
\newline
\mbox{Fonte: Própria}}\label{alg:metodoparalel}
\begin{programma}
\ALGORITHM{inicioComunicacaoParalela(M, s, r)}

\STATE inicializaVoos(M,s,r);
\FOR{$cn$ \FROM $1$ \TO $(s-1)$}\PGlnlabel{forline}
\STATE $dest \GETS (r + cn)$ MOD $s$;
\STATE $c \GETS 0$;
\FOREACH{$t$ \IN $M$}
\STATE $f \GETS ultimoVoo(t)$;
	\IF{f ainda não foi selecionado}
	\STATE $c \GETS c + 1$;
	\STATE solicitaConexão(f, dest);
	\ENDIF
\ENDFOR

\STATE EnviaFinalizaMsg(dest);

\WHILE{Não tiver respondido todas as requisições}
	\STATE $P \GETS RecebeMSG()$;
	\IF{P é uma solicitação de conexão}
		\STATE $f \GETS $selecionaUmVoo(M, P.voo);
		\IF{$f \neq \emptyset$}
			\STATE EfetuaConexão(P, f);
		\ELSE
			\STATE RejeitaConexão(P, f);
		\ENDIF
	\ELSEIF{P é uma aceitação de conexão}
		\STATE ConfiguraLigação(P);
	\ELSEIF{P é uma finalização de mensagens}
		\STATE $F \GETS verdadeiro$;
	\ENDIF 
	
\ENDWHILE
\ENDFOR

\ENDALGORITHM
\end{programma}
\end{figure} 


A linha 1 é utilizada na inicialização dos voos, onde cada processo obtém o
conjunto de voos que irá trabalhar, esse conjunto de voos é definido pela
instância $M$, pelo número de processor que estão executando $s$ e pelo número
do processo corrente $r$. O laço iniciado na linha 2 modela a quantidade de
iterações do algoritmo, que é baseado no número de processos. A variável $cn$ é
o contador de nível e informa para qual vizinho o processo corrente irá tentar
efetuar ligações dos seus trilhos disponíveis.

O cálculo do índice do processo que irá receber a comunicação é feita na linha
3. A linha 4 inicia o contador de mensagens enviadas, que é usado como auxiliar
de condição de parada no laço da linha 13.

É feito então uma tentativa de conectar os trilhos que ainda não
foram conectados com os trilhos dos vizinhos. Nas linhas 6 a 10 é
feito o envio de solicitações de conexões de todos os últimos voos ainda não
conectados de cada um dos seus trilhos para o vizinho corrente que é definido
pela variável $dest$. Na linha 12 o processo sinaliza para o seu vizinho
corrente que não tem mais nenhum trilho para tentar efetuar uma ligação.

Depois o processo passa a agir como uma máquina de estados, como pode ser visto
nas linhas 13 a 27, respondendo as requisições que forem recebendo. Isso
acontece até ele receber uma mensagem de finalização do seu vizinho e todas as mensagens
que ele tenha enviado anteriormente tenham sido respondidas. Na linha 14
o processo recebe uma mensagem que vai lhe informar qual a ação que deverá
ser executada.

Quando a mensagem é uma solicitação de conexão, linhas 15 a 22, então o processo
tenta selecionar um voo, do inicio de um de seus trilhos, que ainda não tenha se
ligado com nenhum outro trilho para se ligar com o voo candidato da solicitação.
A escolha do voo que vai se ligar é feita usando o GRASP com uma LRC. Caso
exista algum voo que possa se conectar, linha 17, então uma mensagem de
aceitação de conexão é enviada para o solicitante e o voo é marcado como já
selecionado, isso é feito no procedimento \textit{EfetuaConexão(P, f)}. Caso
não exista nenhum voo que possa se candidatar então uma mensagem de rejeição de
conexão é enviada pelo procedimento \textit{RejeitaConexão(P, f)}.

As linhas 22 a 24 é executada quando o processo recebe uma mensagem de aceitação
ou rejeição de conexão e é utilizada para configurar o voo que teve a
solicitação aceita. Já as linhas 24 a 26 marca a flag de finalização do vizinho
$F$ como verdadeiro.

Na prática essa estratégia conseguiu obter resultados aproximadamente 10\%
melhores que a estratégia sequencial, ou seja, de um processo com a fatia
completa do tempo, levando em consideração apenas a fase de construção. Por
questões de tempo esse experimento não foi concluído, com a inserção da fase de
refinamento que utiliza o \textit{solver}.

Para justificar o uso de uma estratégia mais eficiente, como essa, seria
necessário obter primeiro uma instância que não obtivesse resultados
satisfatórios com a estratégia híbrida sequencial. Aparentemente essa nova
estratégia iria ocasionar melhoras nos resultados finais já que a qualidade da
solução gerada na construção tem influencia direta na velocidade que o
algoritmo encontra uma melhor solução.



% Dessa forma o
%número de processos que serão usados deverá ser calibrado de acordo com a 
%duração da instância.

%Nos testes realizados constatou-se que fatias de tempos menores que 6 horas não
% conseguiram gerar resultados satisfatórios.
 
   
  \chapter{Resultados}

%	Para se obter um limite inferior dessas instâncias foi feita uma verificação
%com o algoritmo do Anexo X que permite checar a %quantidade mínima de voos que
%colidem em uma determinada janela de tempo que é definida pelo atraso máximo
%permitido. (Pode-se fazer uma %formula para explicar esse funcionamento). Essa
%quantidade é dito como sendo o limite inferior da instância e é garantido que
%não existe %solução com uma melhor quantidade de trilhos que essa sem que
%nenhum vôo seja excluído.

Todos os algoritmos descritos foram desenvolvidas na linguagem C++ usando o
solver CPLEX Academic 12 da IBM que implementa técnicas de resolução de
programação inteira. Todos os experimentos computacionais foram feitos em um
notebook com processador Pentium T4500 2.3 Ghz, 2 GBytes de memória RAM (2x1 GB)
e com o sistema operacional operacional Linux Ubuntu 11.04 de 32 bits.

O GRASP foi configurado com 10 iterações completas do GRASP utilizando um
$\alpha$ de 0.5. Esses valores foram obtidos de forma empírica a partir
dos testes que foram feitos no decorrer do trabalho. A busca local é exaustiva
e finaliza quando não consegue melhorar o valor objetivo da solução,
a maior utilização do poder de processamento do CPU é gasto executando essa
etapa, cerca de 98\% do tempo.

%pode-se colocar uma tabela comparando os valores obtidos para diferentes tipos
% de alpha. Para assim justificar essa escolha. Essa comparação pode ser feita
% com um grafico de convergência em relação a iteração do GRASP (0.25 - 0.5 -
% 0.75) p/ 1 ou 2 instancias

A demonstração da eficiência do algoritimo foi feita baseado na solução ótima
dos problemas que foram obtidas a partir da utilização do modelo
matemático descrito no Capítulo \ref{cap:metodoprop} nas instâncias. A única
instância em que isso não foi possível foi na da TAM estendida onde em um
período de 36 horas (115200 segundos) não foi possível obter nenhuma solução
inteira para o problema. Essa foi a maior instância utilizada no trabalho.

Nesse capítulo $s*$ indica o valor ótimo de uma solução, $s_{m}$ indica a média
dos valores obtidos com todas as execuções do algoritmo, $t_{s*}$ é o
tempo de execução do solver e $t_{s}$ representa o tempo médio de execução do
algoritmo. Por fim $\Delta \%$ (GAP) representa a diferencia em percentual da
média dos valores das soluções obtidas em relação ao valor ótimo da instância,
o seu cálculo é feito com a forma abaixo:

\[  \Delta \% (GAP) = (s - s*)/s* \]

Para os testes foram utilizados duas instâncias diárias, uma da Rio-Sul com 107
voos e outra da TAM com 241 voos. A instância da Rio-Sul foi obtida a partir do
trabalho de \cite{pontes2002} e a da TAM foi obtida através da seleção manual
dos voos através do $site$ da companhia aérea. Com o desenvolvimento do trabalho
essas instâncias passaram a ser resolvidas facilmente pelo sistema. Com a
finalidade de gerar instâncias mais difíceis foi proposto a extensão da
frequência dos voos da instância Rio-Sul e da TAM para uma semana, dessa forma
foi gerado uma instância com 749 voos e outra de 1687. 

Por causa da grande dificuldade de comunicação com as companhias aéreas foi
utilizado um tempo de solo fictício de 20 minutos para todos os aeroportos esse
valor foi escolhido para menter a compatibilidade dos resultados com o trabalho
de \cite{pontes2002}

A resolução dessas instâncias foram parametrizadas levando em consideração dois
cenários. O cenário 1 faz o sequenciamento dos voos sem a permissão de utilizar
nenhum atraso, essa representação é comum nas companhias que não aceitam a
modificação do planejamento inicial. O cenário 2 se utiliza de atrasos
permitindo assim uma maior liberdade na hora da montagem dos trilhos obtendo
assim um melhor aproveitamento da utilização das aeronáves. Os parâmetros
utilizados são detalhados na Tabela \ref{tab:params}.


\begin{table}
\caption{Parametrização dos cenários}\label{tab:params}
\begin{center}


\begin{tabular}{l|rr}
\hline

 & Cenário 1 & Cenário 2 \\
 \hline
 Atraso Maximo & 0 & 10 \\
 Prob. Arc. Tipo 1 & 0.92 & 0.69\\ 
 Prob. Arc. Tipo 2 & 0 & 0.16\\
 Prob. Arc. Tipo 3 & 0.08 & 0.04 \\
 Prob. Arc. Tipo 4 & 0 & 0.01 \\
  
\hline

\end{tabular}
\end{center}
\end{table}




O algoritmo proposto foi executado 100 vezes e apenas a média dos valores da
solução e do tempo foram levados em consideração. As Tabelas \ref{tab:cenario1}
e \ref{tab:cenario2} mostram os resultados obtidos. Para instâncias pequenas a
utilização do solver é suficiente porém percebeu-se que com o aumento do tamanho
da instância o solver leva muito tempo para resolver e uma estratégia híbrida
pode ser o caminho para obter boas soluções em um curto espaço de tempo.

\begin{table}[ht]
\caption{Resultados do cenário 1}\label{tab:cenario1}


\begin{tabular}{l r r r r r}
\hline

Instância 			& $s*$ (trilhos) & $t_{s*}(s)$ & $s_{m}$ (trilhos) & $t_{s}(s)$ &
$\Delta\%$
\\
\hline

Rio Sul 			& 17.138 (17) & 4 s		 	& 17.138 (17) 	& 7 s 		 & 0\\
TAM     			& 35.334 (34) & 26 s	 	& 35.334 (34)	& 36 s 		 & 0\\
Rio Sul Estendida 	& 18.392 (17) & 24192 s	 	& 18.392 (17)	& 64 s 		 & 0\\
TAM Estendida 		& - 		  & 115200 s 	& 49.857 (35)	& 154 s		 & -\\

\hline
\end{tabular}

\end{table}


\begin{table}[ht]
\caption{Resultados do cenário 2}\label{tab:cenario2}


\begin{tabular}{l r r r r r}
\hline

Instância 			& $s*$ (trilhos) & $t_{s*}(s)$ & $s_{m}$ (trilhos) & $t_{s}(s)$ &
$\Delta\%$
\\
\hline

Rio Sul 			& 16.158 (16) & 4 s		 	& 16.158 (16) 	& 7 s 		 & 0\\
TAM     			& 35.015 (34) & 27 s	 	& 35.015 (34)	& 36 s 		 & 0\\
Rio Sul Est. 	& 17.433 (16) & 33001 s	 	& 17.532 (16)	& 65 s 		 &$<$0.01\\ 
TAM Estendida 		& - 		  & 115200 s 	& 48.803 (35)	& 159 s		 & -\\

\hline
\end{tabular}

\end{table}


Pode-se observar que nos dois cenários a solução ótima foi obtida para a
instância da Rio-Sul. Na instância da TAM a solução ótima foi encontrada,
porém, na média o cenário 1 encontrou uma solução bem próxima. Essas duas
instâncias representam um horizonte de tempo de um dia. Na instância da Rio-Sul
estendida que representam uma semana de operação as soluções ficaram em média
0.19 do ótimo para o cenário 1 e 0.18 no cenário 2 com um tempo aproximado de 8
minutos. Para o procedimento programação linear não foram inseridos limites
previamente calculados, de modo que a ferramenta utilizou apenas a relaxação
linear.
  
Alguns ajustes ainda podem melhorar o modelo híbrido para que ele possa se
aproximar mais da solução ótima. A modificação da estrutura a ser otimizada na
busca local pode ser um ponto que ajude a melhorar os resultados, pois a
literatura mostra que esse é um dos pontos mais importantes de uma heurística
híbrida.
 
Uma das grandes dificuldades encontradas no trabalho foi a falta de instâncias
na literatura tornando difícil a comparação de resultados com outras
abordagens. Com isso existe a necessidade de geração de um conjunto de
instâncias e a sua publicação para fins comparativos.
 
Existe ainda a possibilidade de uma implementação paralela que ainda está em
fase de planejamento e que se demonstrar resultados interessantes em tempo
hábil será adicionada a dissertação.
 

%Com a eficiência obtida com o método exato existe uma necessidade de geração de instâncias maiores que possam ser utilizadas parar ajustar e justificar a utilização de uma abordagem mais complexa como o uso de uma metaheurística híbrida.

%Atualmente o método híbrido conseguiu resolver a instância \textit{TAM Estendida} com o tempo de 60s e Custo total de 43344, que parece %ser uma boa solução tendo como base os resultados obtidos com a instância que lhe serviu de base.

%Atualmente existe a necessidade de um melhor ajuste no método híbrido para que ele possa conseguir resultados mais robustos.

%A utilização de uma abordagem exata em conjunto com metaheurísticas está sendo cada vez mais utilizado na literatura. Porém a escolha da estrutura a ser otimizada deve ser bem escolhida para não aumentar demasiadamente a capacidade computacional necessária para resolver o problema.



%Para demonstrar a eficiência em termos de qualidade da solução da metaheurística GILS, realizamos comparações com um  procedimento exato B&B (XPRESS MP, 2004), implementando o modelo STSP apresentado por Lee (1996). Para cada instância da Tabela 1 temos as duas primeiras colunas representando as dimensões das instâncias testadas e as colunas restantes divididas em dois grupos: procedimento B&B e GILS. No caso do procedimento B&B, a coluna z* indica o valor ótimo e a coluna tempo indica o tempo computacional, em segundos, gasto na resolução da instância. Já para o grupo da metaheurística GILS as colunas adicionais, além da coluna tempo, são: iter que indica a iteração onde foi encontrada melhor solução, z que indica o valor obtido pelo GILS e, Δ (gap) que indica a diferença percentual entre as soluções:
%Δ = [ (z – z*) / z* ] x 100.


% % \chapter{Conclusões}
  
  %Os resultados obtidos, apesar de iniciais, indicam que existe um grande potencial na utilização desse conjunto de técnicas. O que ainda %é necessário é a construção ou a obtenção de uma maior quantidade de instâncias para permitir explorar e ajustar melhor diversos pontos do %método proposto.
  
 % A utilização de uma abordagem exata em conjunto com metaheurísticas está sendo cada vez mais utilizado na literatura. Porém a escolha da %estrutura a ser otimizada deve ser bem escolhida para não aumentar demasiadamente a capacidade computacional necessária para resolver o %problema.

%pode-se incluir um cap de trabalhos futuros
 
% Bibliografia o arquivo

\singlespacing
\bibliographystyle{abnt-alf} % Estilo autor-data
\bibliography{bibliografia}    % As referencias deste testo est?o no arquivo modelotese.bib
\ABNTaddcontentsline{toc}{chapter}{Referências bibliográficas}


\anexo

\chapter{Rede de voos da Rio-Sul}\label{anx:netriosul}


\begin{scriptsize}

\begin{longtable}{l c c l}
001 SL166  0 08:00 0 08:31 GYN BSB & & & 055 SL598  0 16:00 0 17:55 CGH POA \\

002 SL155  0 08:32 0 10:33 GYN CGH & & & 056 SL568  0 16:02 0 17:03 CGH PLU \\

003 SL330  0 08:34 0 09:59 CGH VIX & & & 057 SL403  0 16:08 0 17:59 POA CGH \\

004 SL595  0 08:40 0 11:11 POA CGH & & & 058 SL273  0 16:10 0 17:51 SJP CGH \\

005 SL596  0 08:44 0 10:39 CGH POA & & & 059 SL391  0 16:12 0 17:55 FLN CGH \\

006 SL533  0 08:50 0 10:01 BSB PLU & & & 060 JH343  0 16:12 0 18:57 SSA CGH \\

007 SL280  0 08:52 0 09:43 CGH CWB & & & 061 SL481  0 17:00 0 17:45 CGH SDU \\

008 SL587  0 09:00 0 10:01 PLU CGH & & & 062 JH506  0 17:02 0 18:33 CGH BSB \\

009 SL520  0 09:02 0 09:47 SDU PLU & & & 063 SL405  0 17:14 0 19:57 CGH CGH \\

010 JH559  0 09:04 0 10:35 BSB CGH & & & 064 SL569  0 17:32 0 18:33 PLU CGH \\

011 SL406  0 09:06 0 09:55 SDU CGH & & & 065 SL152  0 17:44 0 19:35 CGH GYN \\

012 SL419  0 09:16 0 11:03 FLN CGH & & & 066 JH546  0 17:56 0 19:27 CGH BSB \\

013 SL542  0 09:28 0 10:31 CGH PLU & & & 067 SL572  0 17:58 0 18:59 CGH PLU \\

014 SL408  0 09:30 0 10:21 SDU CGH & & & 068 SL482  0 18:30 0 19:21 SDU CGH \\

015 JH345  0 09:30 0 12:23 SSA CGH & & & 069 JH341  0 18:50 1 00:19 JPA CGH \\

016 SL589  0 09:58 0 11:01 PLU CGH & & & 070 SL592  0 18:54 0 20:45 CGH POA \\

017 JH502  0 10:02 0 11:33 CGH BSB & & & 071 JH507  0 19:00 0 20:31 BSB CGH \\

018 SL281  0 10:08 0 10:59 CWB CGH & & & 072 SL528  0 19:02 0 19:47 SDU PLU \\

019 SL521  0 10:16 0 11:01 PLU SDU & & & 073 JH550  0 19:04 0 20:35 CGH BSB \\

020 SL532  0 10:20 0 11:31 PLU BSB & & & 074 SL537  0 19:22 0 20:33 BSB PLU \\

021 SL510  0 10:22 0 11:13 CGH CWB & & & 075 SL576  0 19:26 0 20:27 CGH PLU \\

022 SL544  0 10:32 0 11:33 CGH PLU & & & 076 SL573  0 19:34 0 20:35 PLU CGH \\

023 SL331  0 10:40 0 12:11 VIX CGH & & & 077 SL599  0 19:46 0 21:41 POA CGH \\

024 SL409  0 11:00 0 11:49 CGH SDU & & & 078 JH547  0 20:00 0 21:31 BSB CGH \\

025 SL543  0 11:00 0 12:03 PLU CGH & & & 079 SL153  0 20:04 0 22:05 GYN CGH \\

026 SL590  0 11:06 0 12:44 CGH POA & & & 080 SL483  0 20:08 0 20:53 CGH SDU \\

027 SL336  0 11:16 0 15:05 CGH REC & & & 081 SL412  0 20:10 0 21:29 CGH FLN \\

028 SL597  0 11:30 0 13:27 POA CGH & & & 082 SL529  0 20:14 0 20:59 PLU SDU \\

029 SL522  0 11:32 0 12:17 SDU PLU & & & 083 SL584  0 20:26 0 21:27 CGH PLU \\

030 SL150  0 11:42 0 13:33 CGH GYN & & & 084 SL514  0 20:30 0 21:21 CGH CWB \\

031 SL511  0 11:50 0 12:41 CWB CGH & & & 085 JH344  0 20:48 0 23:39 CGH SSA \\

032 JH503  0 12:02 0 13:33 BSB CGH & & & 086 SL577  0 21:02 0 22:03 PLU CGH \\

033 SL545  0 12:04 0 13:05 PLU CGH & & & 087 SL536  0 21:04 0 22:13 PLU BSB \\

034 SL470  0 12:28 0 13:19 SDU CGH & & & 088 JH551  0 21:04 0 22:35 BSB CGH \\

035 SL523  0 12:46 0 13:31 PLU SDU & & & 089 SL332  0 21:20 0 22:31 CGH VIX \\

036 JH342  0 13:00 0 15:45 CGH SSA & & & 090 SL492  0 21:30 0 22:21 SDU CGH \\

037 JH500  0 13:02 0 14:33 CGH BSB & & & 091 SL586  0 21:30 0 22:31 CGH PLU \\

038 SL147  0 13:10 0 18:03 BSB BSB & & & 092 SL530  0 21:34 0 22:19 SDU PLU \\

039 JH340  0 13:22 0 18:25 CGH JPA & & & 093 SL515  0 21:45 0 22:37 CWB CGH \\

040 SL591  0 13:30 0 14:57 POA CGH & & & 094 SL593  0 21:52 0 23:19 POA CGH \\

041 SL402  0 13:38 0 15:33 CGH POA & & & 095 SL585  0 22:04 0 23:05 PLU CGH \\

042 SL471  0 13:54 0 14:39 CGH SDU & & & 096 SL518  0 22:16 0 23:07 CGH CWB \\

043 SL390  0 13:56 0 15:41 CGH FLN & & & 097 SL413  0 22:22 0 23:31 FLN CGH \\

044 SL564  0 13:58 0 14:59 CGH PLU & & & 098 SL407  0 22:24 0 23:09 CGH SDU \\

045 SL151  0 13:58 0 15:59 GYN CGH & & & 099 SL167  0 22:42 0 23:13 BSB GYN \\

046 SL563  0 14:00 0 15:01 PLU CGH & & & 100 SL531  0 22:48 0 23:33 PLU SDU \\

047 SL272  0 14:12 0 15:51 CGH SJP & & & 101 JH558  0 22:56 1 00:27 CGH BSB \\

048 SL282  0 14:22 0 15:13 CGH CWB & & & 102 SL493  0 23:02 0 23:47 CGH SDU \\

049 SL508  0 14:56 0 15:41 SDU PLU & & & 103 SL333  0 23:04 1 00:15 VIX CGH \\

050 JH501  0 14:58 0 16:29 BSB CGH & & & 104 SL588  0 23:06 1 00:07 CGH PLU \\

051 SL480  0 15:30 0 16:21 SDU CGH & & & 105 SL594  0 23:18 1 01:57 CGH POA \\

052 SL283  0 15:42 0 16:37 CWB CGH & & & 106 SL154  0 23:22 1 01:13 CGH GYN \\

053 SL337  0 15:46 0 19:29 REC CGH & & & 107 SL418  0 23:32 1 01:19 CGH FLN \\

054 SL347  0 16:00 0 16:45 PLU SDU & & & \\

\end{longtable}

\end{scriptsize}
\chapter{Tempo dos voos da Rio-Sul}\label{anx:timeriosul}

A tabela abaixo está organizada em 3 colunas onde as duas primeiras representam
dois aeroportos distintos e a terceira representa o tempo de voo em minutos
entre esses dois aeroportos.

\begin{scriptsize}

\begin{longtable}{l l l l}


BSB CGH 0091 & CGH SJP 0049 & FLN SSA 0225 & PLU SDU 0045 \\
BSB CWB 0125 & CGH SSA 0171 & FLN VIX 0136 & PLU SJP 0067 \\
BSB FLN 0152 & CGH VIX 0087 & GYN JPA 0216 & PLU SSA 0113 \\
BSB GYN 0031 & CWB FLN 0028 & GYN PLU 0076 & PLU VIX 0044 \\
BSB JPA 0197 & CWB GYN 0114 & GYN POA 0173 & POA REC 0342 \\
BSB PLU 0071 & CWB JPA 0293 & GYN REC 0210 & POA SDU 0129 \\
BSB POA 0185 & CWB PLU 0095 & GYN SDU 0109 & POA SJP 0119 \\
BSB REC 0191 & CWB POA 0062 & GYN SJP 0054 & POA SSA 0267 \\
BSB SDU 0107 & CWB REC 0283 & GYN SSA 0143 & POA VIX 0177 \\
BSB SJP 0066 & CWB SDU 0078 & GYN VIX 0118 & REC SDU 0215 \\
BSB SSA 0125 & CWB SJP 0060 & JPA PLU 0198 & REC SJP 0242 \\
BSB VIX 0109 & CWB SSA 0208 & JPA POA 0352 & REC SSA 0075 \\
CGH CWB 0051 & CWB VIX 0125 & JPA REC 0013 & REC VIX 0169 \\
CGH FLN 0105 & FLN GYN 0142 & JPA SDU 0226 & SDU SJP 0079 \\
CGH GYN 0095 & FLN JPA 0311 & JPA SJP 0251 & SDU SSA 0130 \\
CGH JPA 0255 & FLN PLU 0114 & JPA SSA 0085 & SDU VIX 0048 \\
CGH PLU 0058 & FLN POA 0042 & JPA VIX 0181 & SJP SSA 0169 \\
CGH POA 0097 & FLN REC 0300 & PLU POA 0155 & SJP VIX 0110 \\
CGH REC 0246 & FLN SDU 0087 & PLU REC 0187 & SSA VIX 0097 \\
CGH SDU 0049 & FLN SJP 0088 & & \\


\end{longtable}

\end{scriptsize}
\chapter{Resultado ótimo da instância Rio-Sul}\label{anx:resultriosul}

Valor da função objetivo: 16158

\begin{scriptsize}

\begin{longtable}{l l l}

Rota[01 - 8]  & Rota[02 - 7]  & Rota[03 - 4] \\
SL166  0 08:00 0 08:31 GYN BSB & SL155  0 08:32 0 10:33 GYN CGH & SL330  0 08:34 0 09:59 CGH VIX\\
JH559  0 09:04 0 10:35 BSB CGH & SL409  0 11:00 0 11:49 CGH SDU & SL331  0 10:40 0 12:11 VIX CGH\\
SL590  0 11:06 0 12:44 CGH POA & SL470  0 12:28 0 13:19 SDU CGH & JH340  0 13:22 0 18:25 CGH JPA\\
SL591  0 13:30 0 14:57 POA CGH & SL564  0 13:58 0 14:59 CGH PLU & JH341  0 18:50 1 00:19 JPA CGH\\
SL568  0 16:02 0 17:03 CGH PLU & SL347  0 16:00 0 16:45 PLU SDU & \\
SL569  0 17:32 0 18:33 PLU CGH & SL482  0 18:30 0 19:21 SDU CGH & \\
SL592  0 18:54 0 20:45 CGH POA & JH344  0 20:48 0 23:39 CGH SSA & \\
SL593  0 21:52 0 23:19 POA CGH &  & \\

& & \\

Rota[04 - 7]  & Rota[05 - 7]  & Rota[06 - 5] \\
SL595  0 08:40 0 11:11 POA CGH & SL596  0 08:44 0 10:39 CGH POA & SL533  0 08:50 0 10:01 BSB PLU(-1)\\
JH500  0 13:02 0 14:33 CGH BSB & SL597  0 11:30 0 13:27 POA CGH & SL532  0 10:20 0 11:31 PLU BSB\\
JH501  0 14:58 0 16:29 BSB CGH & SL282  0 14:22 0 15:13 CGH CWB & SL147  0 13:10 0 18:03 BSB BSB\\
SL572  0 17:58 0 18:59 CGH PLU & SL283  0 15:42 0 16:37 CWB CGH & JH507  0 19:00 0 20:31 BSB CGH\\
SL573  0 19:34 0 20:35 PLU CGH & SL152  0 17:44 0 19:35 CGH GYN & SL586  0 21:30 0 22:31 CGH PLU\\
SL332  0 21:20 0 22:31 CGH VIX & SL153  0 20:04 0 22:05 GYN CGH & \\
SL333  0 23:04 1 00:15 VIX CGH & JH558  0 22:56 1 00:27 CGH BSB & \\


& & \\

Rota[07 - 8]  & Rota[08 - 8]  & Rota[09 - 9] \\
SL280  0 08:52 0 09:43 CGH CWB & SL587  0 09:00 0 10:01 PLU CGH & SL520  0 09:02 0 09:47 SDU PLU\\
SL281  0 10:08 0 10:59 CWB CGH & SL544  0 10:32 0 11:33 CGH PLU & SL521  0 10:16 0 11:01 PLU SDU\\
SL150  0 11:42 0 13:33 CGH GYN & SL545  0 12:04 0 13:05 PLU CGH & SL522  0 11:32 0 12:17 SDU PLU\\
SL151  0 13:58 0 15:59 GYN CGH & SL402  0 13:38 0 15:33 CGH POA & SL523  0 12:46 0 13:31 PLU SDU\\
SL405  0 17:14 0 19:57 CGH CGH & SL403  0 16:08 0 17:59 POA CGH & SL508  0 14:56 0 15:41 SDU PLU\\
SL514  0 20:30 0 21:21 CGH CWB & SL483  0 20:08 0 20:53 CGH SDU & \textbf{REPO}   0 16:01 0 16:59 PLU CGH\\
SL515  0 21:45 0 22:37 CWB CGH & SL530  0 21:34 0 22:19 SDU PLU & JH546  0 17:56 0 19:27 CGH BSB\\
SL154  0 23:22 1 01:13 CGH GYN & SL531  0 22:48 0 23:33 PLU SDU & JH547  0 20:00 0 21:31 BSB CGH\\
 &  & SL493  0 23:02 0 23:47 CGH SDU\\


& & \\

Rota[10 - 7]  & Rota[11 - 6]  & Rota[12 - 7] \\
SL406  0 09:06 0 09:55 SDU CGH & SL419  0 09:16 0 11:03 FLN CGH & SL542  0 09:28 0 10:31 CGH PLU\\
SL510  0 10:22 0 11:13 CGH CWB & \textbf{REPO}   0 11:23 0 12:21 CGH PLU & SL543  0 11:00 0 12:03 PLU CGH\\
SL511  0 11:50 0 12:41 CWB CGH & SL563  0 14:00 0 15:01 PLU CGH & SL390  0 13:56 0 15:41 CGH FLN\\
SL272  0 14:12 0 15:51 CGH SJP & SL598  0 16:00 0 17:55 CGH POA & SL391  0 16:12 0 17:55 FLN CGH\\
SL273  0 16:10 0 17:51 SJP CGH(+1) & SL599  0 19:46 0 21:41 POA CGH & JH550  0 19:04 0 20:35 CGH BSB\\
SL412  0 20:10 0 21:29 CGH FLN & SL518  0 22:16 0 23:07 CGH CWB & JH551  0 21:04 0 22:35 BSB CGH\\
SL413  0 22:22 0 23:31 FLN CGH &  & SL594  0 23:18 1 01:57 CGH POA\\

& & \\

Rota[13 - 6]  & Rota[14 - 6]  & Rota[15 - 8] \\
SL408  0 09:30 0 10:21 SDU CGH & JH345  0 09:30 0 12:23 SSA CGH & SL589  0 09:58 0 11:01 PLU CGH\\
SL336  0 11:16 0 15:05 CGH REC & JH342  0 13:00 0 15:45 CGH SSA & SL471  0 13:54 0 14:39 CGH SDU\\
SL337  0 15:46 0 19:29 REC CGH & JH343  0 16:12 0 18:57 SSA CGH & SL480  0 15:30 0 16:21 SDU CGH\\
SL584  0 20:26 0 21:27 CGH PLU & SL576  0 19:26 0 20:27 CGH PLU & SL481  0 17:00 0 17:45 CGH SDU\\
SL585  0 22:04 0 23:05 PLU CGH & SL577  0 21:02 0 22:03 PLU CGH & SL528  0 19:02 0 19:47 SDU PLU\\
SL418  0 23:32 1 01:19 CGH FLN & SL407  0 22:24 0 23:09 CGH SDU & SL529  0 20:14 0 20:59 PLU SDU\\
 &  & SL492  0 21:30 0 22:21 SDU CGH\\
 &  & SL588  0 23:06 1 00:07 CGH PLU\\


\end{longtable}

\end{scriptsize}

\chapter{Rede de voos da TAM}\label{anx:nettam}


\begin{scriptsize}

\begin{longtable}{l c c l}

001 JJ3458 0 00:05 0 01:05 MAB BEL & & & 122 JJ3928 0 13:00 0 14:00 CGH SDU \\

002 JJ3585 0 01:10 0 06:21 RBR BSB & & & 123 JJ3871 0 13:15 0 14:15 BEL MAB \\

003 JJ3585 0 01:10 0 06:21 RBR BSB & & & 124 JJ3365 0 13:15 0 16:00 IOS SDU \\

004 JJ3595 0 01:30 0 06:16 PVH BSB & & & 125 JJ3929 0 13:15 0 14:07 SDU CGH \\

005 JJ3459 0 01:35 0 06:25 BEL CNF & & & 126 JJ3571 0 13:25 0 15:56 CGR BSB \\

006 JJ3459 0 01:35 0 02:35 BEL MAB & & & 127 JJ3038 0 13:25 0 14:29 JOI CGH \\

007 JJ3069 0 01:50 0 02:51 CGB CGR & & & 128 JJ3563 0 13:30 0 18:43 RBR BSB \\

008 JJ3739 0 02:30 0 05:59 FOR BSB & & & 129 JJ3563 0 13:30 0 18:43 RBR BSB \\

009 JJ2101 0 02:40 0 04:10 NAT SSA & & & 130 JJ3930 0 13:30 0 14:27 CGH SDU \\

010 JJ3201 0 02:40 0 07:20 NAT CNF & & & 131 JJ3815 0 13:35 0 15:51 PMW BSB \\

011 JJ3201 0 02:40 0 09:22 NAT CGH & & & 132 JJ3708 0 13:43 0 15:19 CGH BSB \\

012 JJ3065 0 02:50 0 06:22 AJU SDU & & & 133 JJ3931 0 13:45 0 14:35 SDU CGH \\

013 JJ3459 0 03:10 0 06:25 MAB CNF & & & 134 JJ3759 0 13:50 0 14:52 CNF SDU \\

014 JJ3409 0 03:23 0 05:27 BPS CNF & & & 135 JJ3745 0 13:55 0 14:56 SJP CGH \\

015 JJ3069 0 03:24 0 06:24 CGR SDU & & & 136 JJ3120 0 14:00 0 15:02 NVT CGH \\

016 JJ3775 0 03:35 0 06:03 CGR CGH & & & 137 JJ3932 0 14:00 0 15:00 CGH SDU \\

017 JJ3201 0 04:45 0 07:20 SSA CNF & & & 138 JJ3661 0 14:12 0 17:14 IOS CGH \\

018 JJ3201 0 04:45 0 09:22 SSA CGH & & & 139 JJ3933 0 14:16 0 15:14 SDU CGH \\

019 JJ3737 0 05:50 0 06:45 SJP CGH & & & 140 JJ3934 0 14:28 0 15:30 CGH SDU \\

020 JJ4723 0 06:00 0 07:43 BSB CGH & & & 141 JJ3264 0 14:30 0 15:53 CWB SDU \\

021 JJ3597 0 06:00 0 08:42 CGB BSB & & & 142 JJ3935 0 14:45 0 15:44 SDU CGH \\

022 JJ3770 0 06:00 0 07:00 GRU IOS & & & 143 JJ3628 0 14:54 0 16:13 CGH SSA \\

023 JJ3900 0 06:04 0 07:04 CGH SDU & & & 144 JJ4737 0 14:55 0 18:02 CGB CGH \\

024 JJ3100 0 06:05 0 07:15 FLN CGH & & & 145 JJ3936 0 15:00 0 16:00 CGH SDU \\

025 JJ3211 0 06:07 0 07:21 CNF CGH & & & 146 JJ3077 0 15:15 0 19:15 REC GIG \\

026 JJ3901 0 06:15 0 07:07 SDU CGH & & & 147 JJ3077 0 15:15 0 21:11 REC CGH \\

027 JJ3768 0 06:20 0 07:27 LDB CGH & & & 148 JJ3937 0 15:15 0 16:11 SDU CGH \\

028 JJ3119 0 06:24 0 07:35 CGH NVT & & & 149 JJ3107 0 15:17 0 16:20 CGH FLN \\

029 JJ3902 0 06:30 0 07:30 CGH SDU & & & 150 JJ3027 0 15:18 0 17:04 BSB SDU \\

030 JJ3758 0 06:34 0 07:47 SDU CNF & & & 151 JJ3410 0 15:22 0 16:20 GIG VIX \\

031 JJ3903 0 06:45 0 07:49 SDU CGH & & & 152 JJ3938 0 15:30 0 16:32 CGH SDU \\

032 JJ3242 0 06:47 0 07:50 CGH UDI & & & 153 JJ3013 0 15:35 0 16:27 CGH CWB \\

033 JJ3035 0 06:49 0 08:00 CGH JOI & & & 154 JJ3744 0 15:42 0 16:40 CGH SJP \\

034 JJ3370 0 06:52 0 07:49 CGH GIG & & & 155 JJ3826 0 15:43 0 17:19 SDU BSB \\

035 JJ3370 0 06:52 0 10:27 CGH REC & & & 156 JJ3939 0 15:45 0 16:35 SDU CGH \\

036 JJ3732 0 06:52 0 08:12 CGH SSA & & & 157 JJ3652 0 15:57 0 16:40 CGH CGR \\

037 JJ3753 0 07:00 0 08:04 CNF SDU & & & 158 JJ3940 0 16:01 0 17:02 CGH SDU \\

038 JJ3666 0 07:00 0 07:45 SDU IOS & & & 159 JJ3723 0 16:03 0 17:38 BSB CGH \\

039 JJ3904 0 07:00 0 08:02 CGH SDU & & & 160 JJ3941 0 16:15 0 17:11 SDU CGH \\

040 JJ3022 0 07:06 0 08:56 SDU BSB & & & 161 JJ3029 0 16:22 0 17:58 BSB SDU \\

041 JJ3855 0 07:10 0 08:15 BSB CNF & & & 162 JJ3942 0 16:30 0 17:28 CGH SDU \\

042 JJ3023 0 07:12 0 09:02 BSB SDU & & & 163 JJ3857 0 16:41 0 17:52 BSB CNF \\

043 JJ3905 0 07:16 0 08:13 SDU CGH & & & 164 JJ3943 0 16:45 0 17:34 SDU CGH \\

044 JJ3906 0 07:30 0 08:29 CGH SDU & & & 165 JJ3263 0 16:47 0 18:55 SDU POA \\

045 JJ3907 0 07:45 0 08:51 SDU CGH & & & 166 JJ3629 0 16:52 0 20:15 SSA CGH \\

046 JJ3740 0 07:55 0 08:55 CGH SJP & & & 167 JJ3104 0 16:55 0 17:56 FLN CGH \\

047 JJ3740 0 07:55 0 10:05 CGH CGB & & & 168 JJ3267 0 16:56 0 18:30 SDU CWB \\

048 JJ3201 0 08:00 0 09:22 CNF CGH & & & 169 JJ3944 0 16:59 0 18:06 CGH SDU \\

049 JJ3908 0 08:00 0 09:00 CGH SDU & & & 170 JJ3966 0 16:59 0 18:06 CGH SDU \\

050 JJ3130 0 08:12 0 09:35 CGH VIX & & & 171 JJ3137 0 17:00 0 18:28 VIX CGH \\

051 JJ3667 0 08:15 0 11:00 IOS SDU & & & 172 JJ3012 0 17:02 0 17:50 CWB CGH \\

052 JJ3118 0 08:15 0 09:09 NVT CGH & & & 173 JJ3123 0 17:07 0 18:08 CGH NVT \\

053 JJ3909 0 08:15 0 09:15 SDU CGH & & & 174 JJ3945 0 17:15 0 18:08 SDU CGH \\

054 JJ3411 0 08:15 0 09:22 VIX GIG & & & 175 JJ3743 0 17:15 0 18:14 SJP CGH \\

055 JJ3269 0 08:18 0 09:48 GIG CWB & & & 176 JJ3946 0 17:29 0 18:32 CGH SDU \\

056 JJ3385 0 08:30 0 09:38 CNF GIG & & & 177 JJ3947 0 17:44 0 18:34 SDU CGH \\

057 JJ3910 0 08:30 0 09:30 CGH SDU & & & 178 JJ3653 0 17:50 0 20:29 CGR CGH \\

058 JJ3239 0 08:30 0 09:41 UDI CGH & & & 179 JJ3028 0 17:54 0 19:43 SDU BSB \\

059 JJ3032 0 08:40 0 09:52 JOI CGH & & & 180 JJ3109 0 17:58 0 19:10 CGH FLN \\

060 JJ3261 0 08:42 0 10:27 SDU BSB & & & 181 JJ3948 0 18:00 0 19:04 CGH SDU \\

061 JJ3911 0 08:44 0 09:55 SDU CGH & & & 182 JJ3949 0 18:15 0 19:05 SDU CGH \\

062 JJ3911 0 08:44 0 09:55 SDU CGH & & & 183 JJ3767 0 18:23 0 19:25 CGH LDB \\

063 JJ3370 0 08:45 0 10:27 GIG REC & & & 184 JJ3755 0 18:26 0 19:26 CNF SDU \\

064 JJ3756 0 08:49 0 09:55 SDU CNF & & & 185 JJ3950 0 18:29 0 19:30 CGH SDU \\

065 JJ3733 0 08:54 0 12:09 SSA CGH & & & 186 JJ3827 0 18:32 0 20:01 BSB GIG \\

066 JJ3850 0 08:59 0 10:14 CNF BSB & & & 187 JJ3224 0 18:38 0 19:55 CGH CNF \\

067 JJ3912 0 09:00 0 10:00 CGH SDU & & & 188 JJ3951 0 18:46 0 19:37 SDU CGH \\

068 JJ3913 0 09:15 0 10:21 SDU CGH & & & 189 JJ3033 0 18:46 0 19:45 CGH JOI \\

069 JJ3914 0 09:30 0 10:30 CGH SDU & & & 190 JJ3122 0 18:48 0 19:49 NVT CGH \\

070 JJ3740 0 09:35 0 10:05 SJP CGB & & & 191 JJ3966 0 18:48 0 19:15 SDU CFB \\

071 JJ3024 0 09:42 0 11:27 SDU BSB & & & 192 JJ3952 0 19:00 0 20:06 CGH SDU \\

072 JJ3915 0 09:45 0 10:52 SDU CGH & & & 193 JJ3238 0 19:04 0 20:10 CGH UDI \\

073 JJ3025 0 09:52 0 11:37 BSB SDU & & & 194 JJ3712 0 19:12 0 20:58 CGH BSB \\

074 JJ3127 0 10:00 0 11:29 VIX CGH & & & 195 JJ3268 0 19:14 0 20:46 CWB GIG \\

075 JJ3411 0 10:02 0 12:05 GIG POA & & & 196 JJ3953 0 19:15 0 20:15 SDU CGH \\

076 JJ3916 0 10:02 0 11:00 CGH SDU & & & 197 JJ3276 0 19:24 0 20:22 CGH RAO \\

077 JJ3372 0 10:04 0 12:12 CGH SSA & & & 198 JJ3954 0 19:29 0 20:28 CGH SDU \\

078 JJ3660 0 10:04 0 10:58 CGH IOS & & & 199 JJ3563 0 19:30 0 21:05 BSB GRU \\

079 JJ3660 0 10:04 0 12:12 CGH SSA & & & 200 JJ3262 0 19:30 0 21:28 POA SDU \\

080 JJ3212 0 10:06 0 11:10 CGH CNF & & & 201 JJ3955 0 19:44 0 20:38 SDU CGH \\

081 JJ3604 0 10:06 0 11:25 CGH SSA & & & 202 JJ3967 0 19:45 0 20:12 CFB SDU \\

082 JJ3917 0 10:15 0 11:12 SDU CGH & & & 203 JJ3110 0 19:50 0 21:05 FLN CGH \\

083 JJ3053 0 10:16 0 11:37 CGH POA & & & 204 JJ3752 0 19:54 0 20:54 SDU CNF \\

084 JJ3856 0 10:29 0 11:43 CNF BSB & & & 205 JJ3956 0 19:59 0 21:03 CGH SDU \\

085 JJ3820 0 10:30 0 12:20 GIG BSB & & & 206 JJ3764 0 20:05 0 21:11 LDB CGH \\

086 JJ3918 0 10:30 0 11:28 CGH SDU & & & 207 JJ4722 0 20:05 0 21:44 CGH BSB \\

087 JJ3246 0 10:32 0 11:39 CGH UDI & & & 208 JJ3957 0 20:15 0 21:08 SDU CGH \\

088 JJ3266 0 10:36 0 11:57 CWB SDU & & & 209 JJ3226 0 20:19 0 21:36 CGH CNF \\

089 JJ3745 0 10:45 0 13:15 CGB SJP & & & 210 JJ3226 0 20:19 0 22:59 CGH SSA \\

090 JJ3745 0 10:45 0 14:56 CGB CGH & & & 211 JJ3226 0 20:19 1 01:05 CGH NAT \\

091 JJ3919 0 10:45 0 11:48 SDU CGH & & & 212 JJ3034 0 20:20 0 21:17 JOI CGH \\

092 JJ3920 0 11:00 0 12:00 CGH SDU & & & 213 JJ3757 0 20:33 0 21:34 CNF SDU \\

093 JJ3574 0 11:04 0 12:20 BSB RBR & & & 214 JJ3030 0 20:37 0 22:23 SDU BSB \\

094 JJ3825 0 11:12 0 12:50 BSB SDU & & & 215 JJ3959 0 20:45 0 21:37 SDU CGH \\

095 JJ3921 0 11:16 0 12:24 SDU CGH & & & 216 JJ3967 0 20:45 0 21:37 SDU CGH \\

096 JJ3922 0 11:30 0 12:25 CGH SDU & & & 217 JJ3245 0 20:45 0 21:55 UDI CGH \\

097 JJ3660 0 11:33 0 12:12 IOS SSA & & & 218 JJ3031 0 20:46 0 22:05 BSB SDU \\

098 JJ3923 0 11:45 0 12:48 SDU CGH & & & 219 JJ3105 0 20:46 0 21:53 CGH FLN \\

099 JJ3215 0 11:50 0 12:57 CNF CGH & & & 220 JJ3958 0 21:00 0 21:55 CGH SDU \\

100 JJ3039 0 11:50 0 12:50 CGH JOI & & & 221 JJ3275 0 21:02 0 22:04 RAO CGH \\

101 JJ3364 0 12:00 0 12:45 SDU IOS & & & 222 JJ3961 0 21:16 0 22:15 SDU CGH \\

102 JJ3605 0 12:00 0 15:34 SSA CGH & & & 223 JJ3854 0 21:28 0 22:44 CNF BSB \\

103 JJ3924 0 12:00 0 13:00 CGH SDU & & & 224 JJ3960 0 21:29 0 22:19 CGH SDU \\

104 JJ3754 0 12:04 0 13:03 SDU CNF & & & 225 JJ3769 0 21:42 0 22:40 CGH LDB \\

105 JJ3260 0 12:06 0 13:31 BSB SDU & & & 226 JJ3736 0 21:56 0 22:57 CGH SJP \\

106 JJ3243 0 12:10 0 13:11 UDI CGH & & & 227 JJ3077 0 22:02 0 23:11 GIG CGH \\

107 JJ3925 0 12:15 0 13:14 SDU CGH & & & 228 JJ3064 0 22:07 0 23:40 SDU AJU \\

108 JJ3265 0 12:19 0 13:45 SDU CWB & & & 229 JJ3774 0 22:07 0 22:47 CGH CGR \\

109 JJ3570 0 12:21 0 12:50 BSB CGR & & & 230 JJ3068 0 22:09 0 23:20 SDU CGR \\

110 JJ3052 0 12:22 0 13:48 POA CGH & & & 231 JJ3226 0 22:20 0 22:59 CNF SSA \\

111 JJ3121 0 12:22 0 13:25 CGH NVT & & & 232 JJ3226 0 22:20 1 01:05 CNF NAT \\

112 JJ3926 0 12:30 0 13:33 CGH SDU & & & 233 JJ3177 0 22:20 0 23:35 GRU FLN \\

113 JJ3814 0 12:41 0 12:55 BSB PMW & & & 234 JJ3458 0 22:25 0 23:35 CNF MAB \\

114 JJ3026 0 12:43 0 14:28 SDU BSB & & & 235 JJ3458 0 22:25 1 01:05 CNF BEL \\

115 JJ3927 0 12:45 0 13:41 SDU CGH & & & 236 JJ3584 0 23:11 1 00:20 BSB RBR \\

116 JJ3410 0 12:47 0 14:41 POA GIG & & & 237 JJ3594 0 23:26 1 00:25 BSB PVH \\

117 JJ3373 0 12:52 0 17:14 SSA CGH & & & 238 JJ3226 0 23:35 1 01:05 SSA NAT \\

118 JJ3661 0 12:52 0 13:37 SSA IOS & & & 239 JJ3738 0 23:47 1 01:20 BSB FOR \\

119 JJ3661 0 12:52 0 17:14 SSA CGH & & & 240 JJ3408 0 23:50 1 00:59 CNF BPS \\

120 JJ3963 0 12:58 0 14:01 SDU CGH & & & 241 JJ3068 0 23:55 1 00:55 CGR CGB \\

121 JJ3709 0 13:00 0 14:38 BSB CGH & & & \\


\end{longtable}

\end{scriptsize}

\chapter{Tempo dos voos da TAM}\label{anx:timetam}


A tabela abaixo está organizada em 3 colunas onde as duas primeiras representam
dois aeroportos distintos e a terceira representa o tempo de voo em minutos
entre esses aeroportos.

\begin{scriptsize}

\begin{longtable}{l l l l}



AJU SDU 0212 & BSB SDU 0110 & CGH NVT 0071 & CNF SDU 0073 \\
BEL CNF 0290 & CFB SDU 0027 & CGH POA 0086 & CNF SSA 0155 \\
BEL MAB 0060 & CGB CGH 0251 & CGH RAO 0062 & CWB GIG 0092 \\
BPS CNF 0124 & CGB CGR 0061 & CGH REC 0356 & CWB SDU 0094 \\
BSB CGB 0162 & CGB SJP 0150 & CGH SDU 0071 & FLN GRU 0075 \\
BSB CGH 0106 & CGH CGR 0159 & CGH SJP 0061 & GIG POA 0123 \\
BSB CGR 0151 & CGH CNF 0082 & CGH SSA 0277 & GIG REC 0240 \\
BSB CNF 0076 & CGH CWB 0052 & CGH UDI 0071 & GIG VIX 0067 \\
BSB FOR 0209 & CGH FLN 0075 & CGH VIX 0089 & GRU IOS 0060 \\
BSB GIG 0110 & CGH GIG 0069 & CGR SDU 0180 & IOS SDU 0165 \\
BSB GRU 0095 & CGH IOS 0182 & CNF GIG 0068 & IOS SSA 0045 \\
BSB PMW 0136 & CGH JOI 0072 & CNF MAB 0195 & NAT SSA 0090 \\
BSB PVH 0286 & CGH LDB 0067 & CNF NAT 0280 & POA SDU 0128 \\
BSB RBR 0313 & CGH NAT 0402 & & \\


\end{longtable}

*O tempo de voos entre alguns aeroportos foram omitidos por falta de informação suficiente para inferi-las.
\end{scriptsize}
\chapter{Resultado ótimo da instância TAM}\label{anx:resulttam}


O resultado está organizado em vários blocos que representam as rotas que foram
obtidas. Cada bloco inicia com um número que identifica a ordem em que a rota se
encontra na lista, seguido de um número que identifica a quantidade de voos
presentes nesta rota. Em seguida é descrito cada voo da rota seguindo o mesmo
padrão utilizado na descrição da instância. Os voos de reposicionamento se
encontram destacados em itálico.

Valor da função objetivo: 35015

\begin{scriptsize}

\begin{longtable}{l l l}

Rota[01 - 8]  & Rota[02 - 5]  & Rota[03 - 8] \\
JJ3458 0 00:05 0 01:05 MAB BEL & JJ3585 0 01:10 0 06:21 RBR BSB & JJ3585 0 01:10 0 06:21 RBR BSB\\
JJ3459 0 01:35 0 06:25 BEL CNF & JJ3855 0 07:10 0 08:15 BSB CNF & JJ3023 0 07:12 0 09:02 BSB SDU\\
\textit{REPO}   0 06:45 0 07:58 CNF SDU & JJ3850 0 08:59 0 10:14 CNF BSB &
JJ3915 0 09:45 0 10:52 SDU CGH\\ 
JJ3909 0 08:15 0 09:15 SDU CGH(+3) & JJ3574 0 11:04 0 12:20 BSB RBR & JJ3922 0
11:30 0 12:25 CGH SDU\\ 
JJ3660 0 10:04 0 12:12 CGH SSA & JJ3563 0 13:30 0 18:43 RBR BSB & JJ3963 0 12:58 0 14:01 SDU CGH\\
JJ3661 0 12:52 0 13:37 SSA IOS & & JJ3628 0 14:54 0 16:13 CGH SSA\\
JJ3661 0 14:12 0 17:14 IOS CGH & & JJ3629 0 16:52 0 20:15 SSA CGH\\
JJ3105 0 20:46 0 21:53 CGH FLN & & JJ3958 0 21:00 0 21:55 CGH SDU\\

\\

Rota[04 - 4]  & Rota[05 - 9]  & Rota[06 - 10] \\
JJ3595 0 01:30 0 06:16 PVH BSB & JJ3459 0 01:35 0 02:35 BEL MAB & JJ3069 0 01:50 0 02:51 CGB CGR\\
\textit{REPO}   0 06:36 0 08:26 BSB GIG & JJ3459 0 03:10 0 06:25 MAB CNF &
JJ3069 0 03:24 0 06:24 CGR SDU\\ 
JJ3370 0 08:45 0 10:27 GIG REC(+1) & JJ3753 0 07:00 0 08:04 CNF SDU & JJ3903 0
06:45 0 07:49 SDU CGH\\ 
JJ3077 0 15:15 0 21:11 REC CGH & JJ3261 0 08:42 0 10:27 SDU BSB & JJ3130 0 08:12 0 09:35 CGH VIX\\
 & JJ3825 0 11:12 0 12:50 BSB SDU & JJ3127 0 10:00 0 11:29 VIX CGH\\
 & JJ3929 0 13:15 0 14:07 SDU CGH & JJ3039 0 11:50 0 12:50 CGH JOI\\
 & JJ3744 0 15:42 0 16:40 CGH SJP & JJ3038 0 13:25 0 14:29 JOI CGH\\
 & JJ3743 0 17:15 0 18:14 SJP CGH & JJ3942 0 16:30 0 17:28 CGH SDU\\
 & JJ3769 0 21:42 0 22:40 CGH LDB & JJ3028 0 17:54 0 19:43 SDU BSB\\
 &  & JJ3031 0 20:46 0 22:05 BSB SDU\\

\\

Rota[07 - 8]  & Rota[08 - 6]  & Rota[09 - 10] \\
JJ3739 0 02:30 0 05:59 FOR BSB & JJ2101 0 02:40 0 04:10 NAT SSA & JJ3201 0 02:40 0 07:20 NAT CNF\\
\textit{REPO}   0 06:19 0 08:09 BSB SDU & JJ3201 0 04:45 0 09:22 SSA CGH &
JJ3201 0 08:00 0 09:22 CNF CGH\\ 
JJ3911 0 08:44 0 09:55 SDU CGH & JJ3604 0 10:06 0 11:25 CGH SSA & JJ3212 0 10:06 0 11:10 CGH CNF\\
JJ3053 0 10:16 0 11:37 CGH POA & JJ3605 0 12:00 0 15:34 SSA CGH & JJ3215 0 11:50 0 12:57 CNF CGH\\
JJ3052 0 12:22 0 13:48 POA CGH & JJ3652 0 15:57 0 16:40 CGH CGR & JJ3930 0 13:30 0 14:27 CGH SDU\\
JJ3940 0 16:01 0 17:02 CGH SDU & JJ3653 0 17:50 0 20:29 CGR CGH & JJ3939 0 15:45 0 16:35 SDU CGH\\
JJ3752 0 19:54 0 20:54 SDU CNF & & JJ3123 0 17:07 0 18:08 CGH NVT\\
JJ3854 0 21:28 0 22:44 CNF BSB & & JJ3122 0 18:48 0 19:49 NVT CGH\\
 &  & JJ3226 0 20:19 0 21:36 CGH CNF\\
 &  & JJ3226 0 22:20 1 01:05 CNF NAT\\

\\

Rota[10 - 5]  & Rota[11 - 8]  & Rota[12 - 8] \\
JJ3201 0 02:40 0 09:22 NAT CGH & JJ3065 0 02:50 0 06:22 AJU SDU & JJ3409 0 03:23 0 05:27 BPS CNF\\
JJ3372 0 10:04 0 12:12 CGH SSA & JJ3666 0 07:00 0 07:45 SDU IOS & JJ3211 0 06:07 0 07:21 CNF CGH\\
JJ3661 0 12:52 0 17:14 SSA CGH & JJ3667 0 08:15 0 11:00 IOS SDU & JJ3908 0 08:00 0 09:00 CGH SDU\\
JJ3948 0 18:00 0 19:04 CGH SDU & JJ3925 0 12:15 0 13:14 SDU CGH & JJ3024 0 09:42 0 11:27 SDU BSB\\
JJ3967 0 20:45 0 21:37 SDU CGH & JJ3708 0 13:43 0 15:19 CGH BSB & JJ3814 0 12:41 0 12:55 BSB PMW\\
 & JJ3723 0 16:03 0 17:38 BSB CGH & JJ3815 0 13:35 0 15:51 PMW BSB\\
 & JJ3954 0 19:29 0 20:28 CGH SDU & JJ3029 0 16:22 0 17:58 BSB SDU\\
 & JJ3961 0 21:16 0 22:15 SDU CGH & JJ3955 0 19:44 0 20:38 SDU CGH\\

\\


Rota[13 - 8]  & Rota[14 - 8]  & Rota[15 - 10] \\
JJ3775 0 03:35 0 06:03 CGR CGH & JJ3201 0 04:45 0 07:20 SSA CNF & JJ3737 0 05:50
0 06:45 SJP CGH(-5)\\ 
JJ3242 0 06:47 0 07:50 CGH UDI & \textit{REPO}   0 07:40 0
08:53 CNF SDU & JJ3904 0 07:00 0 08:02 CGH SDU\\ 
JJ3239 0 08:30 0 09:41 UDI CGH & JJ3913 0 09:15 0 10:21 SDU CGH & JJ3911 0 08:44 0 09:55 SDU CGH\\
JJ3660 0 10:04 0 10:58 CGH IOS & JJ3920 0 11:00 0 12:00 CGH SDU & JJ3918 0 10:30 0 11:28 CGH SDU\\
JJ3660 0 11:33 0 12:12 IOS SSA & JJ3026 0 12:43 0 14:28 SDU BSB & JJ3754 0 12:04 0 13:03 SDU CNF\\
JJ3373 0 12:52 0 17:14 SSA CGH & JJ3027 0 15:18 0 17:04 BSB SDU & JJ3759 0 13:50 0 14:52 CNF SDU\\
JJ3238 0 19:04 0 20:10 CGH UDI & JJ3949 0 18:15 0 19:05 SDU CGH & JJ3826 0 15:43 0 17:19 SDU BSB\\
JJ3245 0 20:45 0 21:55 UDI CGH & JJ3226 0 20:19 0 22:59 CGH SSA & JJ3827 0 18:32 0 20:01 BSB GIG\\
 &  & \textit{REPO}   0 20:21 0 21:29 GIG CNF\\
 &  & JJ3458 0 22:25 0 23:35 CNF MAB\\

\\

Rota[16 - 8]  & Rota[17 - 6]  & Rota[18 - 7] \\
JJ4723 0 06:00 0 07:43 BSB CGH & JJ3597 0 06:00 0 08:42 CGB BSB & JJ3770 0 06:00 0 07:00 GRU IOS\\
JJ3910 0 08:30 0 09:30 CGH SDU & JJ3570 0 12:21 0 12:50 BSB CGR & \textit{REPO}  
0 07:20 0 10:05 IOS SDU\\ 
JJ3917 0 10:15 0 11:12 SDU CGH & JJ3571 0 13:25 0 15:56 CGR BSB & JJ3923 0 11:45 0 12:48 SDU CGH\\
JJ3924 0 12:00 0 13:00 CGH SDU & JJ3857 0 16:41 0 17:52 BSB CNF & JJ3934 0 14:28 0 15:30 CGH SDU\\
JJ3931 0 13:45 0 14:35 SDU CGH & JJ3755 0 18:26 0 19:26 CNF SDU & JJ3943 0 16:45 0 17:34 SDU CGH\\
JJ3224 0 18:38 0 19:55 CGH CNF & JJ3957 0 20:15 0 21:08 SDU CGH & JJ4722 0 20:05 0 21:44 CGH BSB\\
JJ3757 0 20:33 0 21:34 CNF SDU & & JJ3584 0 23:11 1 00:20 BSB RBR\\
JJ3064 0 22:07 0 23:40 SDU AJU\\

\\

Rota[19 - 9]  & Rota[20 - 9]  & Rota[21 - 12] \\
JJ3900 0 06:04 0 07:04 CGH SDU(-8) & JJ3100 0 06:05 0 07:15 FLN CGH & JJ3901 0
06:15 0 07:07 SDU CGH\\ 
JJ3905 0 07:16 0 08:13 SDU CGH & JJ3740 0 07:55 0 10:05 CGH CGB & JJ3906 0 07:30 0 08:29 CGH SDU\\
JJ3912 0 09:00 0 10:00 CGH SDU & JJ3745 0 10:45 0 13:15 CGB SJP & JJ3756 0 08:49 0 09:55 SDU CNF\\
JJ3919 0 10:45 0 11:48 SDU CGH & JJ3745 0 13:55 0 14:56 SJP CGH & JJ3856 0 10:29 0 11:43 CNF BSB\\
JJ3121 0 12:22 0 13:25 CGH NVT & JJ3107 0 15:17 0 16:20 CGH FLN & JJ3260 0 12:06 0 13:31 BSB SDU\\
JJ3120 0 14:00 0 15:02 NVT CGH & JJ3104 0 16:55 0 17:56 FLN CGH & JJ3933 0 14:16 0 15:14 SDU CGH\\
JJ3946 0 17:29 0 18:32 CGH SDU & JJ3950 0 18:29 0 19:30 CGH SDU & JJ3013 0 15:35 0 16:27 CGH CWB\\
JJ3953 0 19:15 0 20:15 SDU CGH & JJ3030 0 20:37 0 22:23 SDU BSB & JJ3012 0 17:02 0 17:50 CWB CGH\\
JJ3960 0 21:29 0 22:19 CGH SDU & JJ3594 0 23:26 1 00:25 BSB PVH & JJ3767 0 18:23 0 19:25 CGH LDB\\
 &  & JJ3764 0 20:05 0 21:11 LDB CGH\\
 &  & JJ3774 0 22:07 0 22:47 CGH CGR\\
 &  & JJ3068 0 23:55 1 00:55 CGR CGB\\

\\

Rota[22 - 7]  & Rota[23 - 8]  & Rota[24 - 9] \\
JJ3768 0 06:20 0 07:27 LDB CGH & JJ3119 0 06:24 0 07:35 CGH NVT & JJ3902 0 06:30
0 07:30 CGH SDU(-5)\\
JJ3740 0 07:55 0 08:55 CGH SJP & JJ3118 0 08:15 0 09:09 NVT CGH & JJ3907 0 07:45 0 08:51 SDU CGH\\
JJ3740 0 09:35 0 10:05 SJP CGB & JJ3914 0 09:30 0 10:30 CGH SDU & JJ3916 0 10:02 0 11:00 CGH SDU\\
JJ4737 0 14:55 0 18:02 CGB CGH & JJ3921 0 11:16 0 12:24 SDU CGH & JJ3364 0 12:00 0 12:45 SDU IOS\\
JJ3033 0 18:46 0 19:45 CGH JOI & JJ3928 0 13:00 0 14:00 CGH SDU & JJ3365 0 13:15 0 16:00 IOS SDU\\
JJ3034 0 20:20 0 21:17 JOI CGH & JJ3937 0 15:15 0 16:11 SDU CGH & JJ3263 0 16:47 0 18:55 SDU POA\\
JJ3736 0 21:56 0 22:57 CGH SJP & JJ3109 0 17:58 0 19:10 CGH FLN & JJ3262 0 19:30 0 21:28 POA SDU\\
 & JJ3110 0 19:50 0 21:05 FLN CGH & \textit{REPO} 0 21:48 0 23:01 SDU CNF\\
 &  & JJ3408 0 23:50 1 00:59 CNF BPS\\

\\

Rota[25 - 8]  & Rota[26 - 8]  & Rota[27 - 8] \\
JJ3758 0 06:34 0 07:47 SDU CNF & JJ3035 0 06:49 0 08:00 CGH JOI & JJ3370 0 06:52 0 07:49 CGH GIG\\
JJ3385 0 08:30 0 09:38 CNF GIG & JJ3032 0 08:40 0 09:52 JOI CGH & JJ3269 0 08:18 0 09:48 GIG CWB\\
JJ3820 0 10:30 0 12:20 GIG BSB & JJ3246 0 10:32 0 11:39 CGH UDI & JJ3266 0 10:36 0 11:57 CWB SDU\\
JJ3709 0 13:00 0 14:38 BSB CGH & JJ3243 0 12:10 0 13:11 UDI CGH & JJ3927 0 12:45 0 13:41 SDU CGH\\
JJ3944 0 16:59 0 18:06 CGH SDU & JJ3932 0 14:00 0 15:00 CGH SDU & JJ3936 0 15:00 0 16:00 CGH SDU\\
JJ3951 0 18:46 0 19:37 SDU CGH & JJ3947 0 17:44 0 18:34 SDU CGH & JJ3267 0 16:56 0 18:30 SDU CWB\\
JJ3956 0 19:59 0 21:03 CGH SDU & JJ3712 0 19:12 0 20:58 CGH BSB & JJ3268 0 19:14 0 20:46 CWB GIG\\
JJ3068 0 22:09 0 23:20 SDU CGR & JJ3738 0 23:47 1 01:20 BSB FOR & JJ3077 0 22:02 0 23:11 GIG CGH\\

\\

Rota[28 - 4]  & Rota[29 - 10]  & Rota[30 - 7] \\
JJ3370 0 06:52 0 10:27 CGH REC & JJ3732 0 06:52 0 08:12 CGH SSA & JJ3022 0 07:06 0 08:56 SDU BSB\\
JJ3077 0 15:15 0 19:15 REC GIG & JJ3733 0 08:54 0 12:09 SSA CGH & JJ3025 0 09:52 0 11:37 BSB SDU\\
\textit{REPO}   0 19:35 0 20:43 GIG CNF & JJ3926 0 12:30 0 13:33 CGH SDU &
JJ3265 0 12:19 0 13:45 SDU CWB\\
JJ3458 0 22:25 1 01:05 CNF BEL & JJ3935 0 14:45 0 15:44 SDU CGH & JJ3264 0 14:30 0 15:53 CWB SDU\\
 & JJ3966 0 16:59 0 18:06 CGH SDU & JJ3941 0 16:15 0 17:11 SDU CGH\\
 & JJ3966 0 18:48 0 19:15 SDU CFB & JJ3952 0 19:00 0 20:06 CGH SDU\\
 & JJ3967 0 19:45 0 20:12 CFB SDU & JJ3959 0 20:45 0 21:37 SDU CGH\\
 & \textit{REPO}   0 20:32 0 21:45 SDU CNF &\\
 & JJ3226 0 22:20 0 22:59 CNF SSA &\\
 & JJ3226 0 23:35 1 01:05 SSA NAT &\\

\\

Rota[31 - 6]  & Rota[32 - 5]  & Rota[33 - 1] \\
JJ3411 0 08:15 0 09:22 VIX GIG & JJ3745 0 10:45 0 14:56 CGB CGH & JJ3871 0 13:15 0 14:15 BEL MAB\\
JJ3411 0 10:02 0 12:05 GIG POA & JJ3938 0 15:30 0 16:32 CGH SDU\\
JJ3410 0 12:47 0 14:41 POA GIG & JJ3945 0 17:15 0 18:08 SDU CGH\\
JJ3410 0 15:22 0 16:20 GIG VIX & JJ3276 0 19:24 0 20:22 CGH RAO\\
JJ3137 0 17:00 0 18:28 VIX CGH & JJ3275 0 21:02 0 22:04 RAO CGH\\
JJ3226 0 20:19 1 01:05 CGH NAT\\

\\

Rota[34 - 3]  \\
JJ3563 0 13:10 0 18:43 RBR BSB\\
JJ3563 0 19:10 0 21:05 BSB GRU\\
JJ3177 0 22:00 0 23:35 GRU FLN\\

\end{longtable}

\end{scriptsize}  

\end{document} 

