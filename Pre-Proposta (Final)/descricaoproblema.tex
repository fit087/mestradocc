\chapter{Descrição do problema}\label{cap:descprob}

%%%Aqui parece ser descrição do problema
  Os voos são organizados de forma que cada aeronave seja responsável por uma sequência válida, que é chamada de trilho.  O conjunto desses trilhos é a solução do problema e é denominada de malha. Essa malha deve conter o menor número possível de trilhos que atenda todos os voos planejados com o mínimo de modificações.
  
  As restrições que envolvem o PCTA induzem a formação de uma rede de possíveis conexões. Nessa rede os nós representam os voos e os arcos representam as conexões possíveis entre esses voos. Dessa forma o problema pode ser formulado como um problema de minimização de custos em uma rede.
  
  Dado a possibilidade de mudanças no tempo de partida sugerido dos voos e também a permissão para criar voos de
reposicionamento, uma grande quantidade de soluções podem ser geradas. A ligação dos voos pode ocorrer de 6 formas
distintas aqui denominado tipos de arcos. Os arcos do tipo 1 permitem a ligação de voos sem a utilização de atrasos e/ou
reposicionamento. Os arcos do tipo 2 utilizam atrasos mas não o reposicionamento. Os arcos do tipo 3 permitem o
sequenciamento com a utilização de um voo de reposicionamento mas sem inserir atraso em nenhum dos voos envolvidos. Os
arcos do tipo 4 utilizam-se de atrasos e de um voo de reposicionamento para fazer a ligação entre dois voos. Os arcos do tipo 5
e 6 são usados no modelo, que é baseado no fluxo em grafo, para representar respectivamente o nó origem(source) e o
destino(sink). Abaixo um maior detalhamento desses arcos:

  
  \begin{itemize}
\item Conexão natural (Arco do tipo 1) - Os arcos desse tipo conectam dois voos respeitando o tempo de partida sugerido e
a restrição geográfica. Eles são associados com as ligações que não requerem mudanças no tempo de partida e nem
       precisam de voos de reposicionamento. O arco do tipo 1 não apresenta custo para ser adicionado a solução.

\item Mudança no tempo (Arco do tipo 2) - Apesar de ter os voos incidentes no mesmo aeroporto, os arcos desse tipo não
permitem a ligação de forma direta pois o tempo de solo disponível não é suficiente para permitir a ligação. No
      entanto, a escolha desse tipo de arco implica em uma mudança no tempo de partida sugerido para quaisquer um dos
      voos envolvidos. O custo de um arco desse tipo é igual a soma dos valores absolutos dos atrasos (em minutos) dos
      horários de partida envolvidos.

\item Voos de reposicionamento (Arco do tipo 3) - Esses arcos representam conexões entre dois voos em que a origem
parte de um aeroporto diferente do local de partida do voo de destino, no entanto, existe tempo suficiente para um voo
de reposicionamento, entre os dois locais, sem violar as restrições de tempo de solo. Os custos de um arco do tipo 3 é
  igual a duração do voo de reposicionamento, incluindo o seu tempo de solo.

\item Voos de reposicionamento mais mudança de tempo (Arco do tipo 4) - Esses arcos representam conexões que
precisam de um voo de reposicionamento mais mudança no tempo de partida sugerido. O arco do tipo 4 tem custo
igual tempo do voo de reposicionamento, incluindo o tempo de solo, mais a soma dos atrasos dos horários de partida
envolvidos em valor absoluto.

\item Arcos do nó fonte ou \textit{source} (Arco do tipo 5) - Esses, arcos são criados para identificar o inicio de um trilho e é com
 ele também que se sabe a quantidade de trilhos necessários para resolver o problema. Cada arco do tipo 5 tem o custo
  igual a 1000.

\item Arcos do nó final ou \textit{sink} (Arco do tipo 6) - Esses arcos tem como destino o nó fictício que é usado para finalizar um
  trilho no modelo. Os arcos do tipo 6 não tem custo.


\end{itemize}
  
%%%% Até aqui %%%%%%%%%%